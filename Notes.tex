%TODO的使用
%%TODO 有关内容
%%FIXME 格式的bug
%%HACK 粗鄙技巧

%TODO——注意事项
%FIXME:20160422 en dash 与连接号 hyphen (人名中的使用)
%% 脚注段落缩进,带框公式不加标点
%% 理想气体、可逆、绝热等公式特殊标注
%% 绝热:Adiabatic 等压:Isobaric 等容:Isochoric 等温:Isothermal 等熵:Isentropic 等焓:Isenthalpic 多方:Polytropic 准静态:Quasistatic

\documentclass[UTF8]{ctexbook}

%TODO——宏包
%页面格式
\usepackage{geometry}
	\geometry{a4paper, twoside, left = 2.54 cm, right = 2.54 cm, top = 3.18 cm, bottom = 3.18 cm, headheight = 3 cm}
\usepackage{fancyhdr}
	\pagestyle{fancy}
	\fancyhf{}
	\fancyhead[EL]{\nouppercase{\CJKfamily{楷体} \leftmark}}
	\fancyhead[OR]{\nouppercase{\CJKfamily{楷体} \rightmark}}
	\fancyfoot[C]{\thepage}
	\renewcommand{\headrulewidth}{0 pt}

\usepackage{titlesec}
%	\newcommand{\sectionbreak}{\clearpage}

%交叉引用、超链接、PDF书签
\usepackage[hyperfootnotes=false]{hyperref}
	\hypersetup{
		bookmarksopen = true,
		bookmarksopenlevel = 1,
		pdfborder = 0 0 0
	}%FIXME:PDF书签有公式会报错,Latex入门 p173

%脚注
\usepackage[stable, perpage]{footmisc}
\usepackage{pifont}
	\renewcommand{\thefootnote}{\ding{\numexpr171+\value{footnote}}}
%\interfootnotelinepenalty = 10000

%数学支持
\usepackage{amsmath}
	\numberwithin{equation}{section}
	\renewcommand{\theequation}{\thesection.\arabic{equation}} %公式编号样式x.x.x
\usepackage{mathtools}
\usepackage[adobe-garamond]{mathdesign}
%\usepackage[urw-garamond]{mathdesign} %免费字体
\usepackage[ntheorem]{empheq}
\usepackage{bm}
%\usepackage{pifont} %已在“脚注”中调用
	\newcommand{\cmark}{\ding{51}}
	\newcommand{\xmark}{\ding{55}}

%中文支持
\usepackage{xeCJK}
	\xeCJKsetup{
%		PunctStyle = quanjiao,
		AutoFakeBold = false,
		AutoFakeSlant = false
	}
	\setCJKmainfont[BoldFont = 华文中宋, ItalicFont = 华文楷体, Mapping = fullwidth-stop]{华文宋体}
	\setCJKfamilyfont{中宋}[Mapping = fullwidth-stop]{华文中宋}
	\setCJKfamilyfont{楷体}[Mapping = fullwidth-stop]{华文楷体}
	\setCJKfamilyfont{仿宋}[Mapping = fullwidth-stop]{华文仿宋}
	\setCJKfamilyfont{黑体}[Mapping = fullwidth-stop]{华文黑体}
	\newcommand{\myHeavy}{\fontspec{MyriadPro-Regular} \CJKfamily{黑体}}
%\usepackage{ctex} %已经使用了ctexbook文档类
	\setcounter{secnumdepth}{4}
	\ctexset{
		part = {
			format = {\bfseries \Huge \centering},
			name = {第, 篇}
		},
		chapter = {
			format = {\bfseries \LARGE \raggedright},
		},
		section = {
			format = {\bfseries \Large \centering},
		},
		subsection = {
			format = {\bfseries},
			name = {,、\hspace{-1 em}},
			numbering = true,
			number = \chinese{subsection},
		}
	}

%符号支持
%\usepackage{gensymb}

%单位
\usepackage{siunitx}
	\sisetup{
		number-math-rm = \ensuremath,
		inter-unit-product = \ensuremath{{}\cdot{}},
		group-digits = integer,
		group-minimum-digits = 4,
		group-separator = \text{~}
	}

%枚举
\usepackage{enumitem}
	\SetLabelAlign{leftalignwithindent}{\hspace{2.1 em} \makebox[1.6 em][l]{#1}}

%定理类环境
\usepackage[thmmarks, amsmath]{ntheorem}
	\theoremstyle{empty} %不带编号
	\theoremsymbol{\ensuremath{\Box}}
		\newtheorem{_myProof}{}
	
	\theoremstyle{empty} %不带编号
	\theoremheaderfont{\myHeavy}
	\theorembodyfont{\CJKfamily{楷体}}
	\theoremsymbol{}
%	\theoremindent \parindent
		\newtheorem{_myThm}{}
	
	\theoremstyle{plain} %带编号
	\theoremheaderfont{\myHeavy}
	\theorembodyfont{\normalfont}
	\theoremsymbol{}
%	\theoremsymbol{\ensuremath{\triangleleft}}
		\newtheorem{myExample}{例}[chapter]

%表格
\usepackage{array}
	\newcolumntype{M}{>{$}c<{$}} %数学模式,居中
\usepackage{tabularx}
	\newcolumntype{Y}{>{\centering\arraybackslash}X} %定宽居中
\usepackage{booktabs} %三线表

%图片
\usepackage{graphicx}
%\usepackage{svg}
\usepackage{asymptote}
	\begin{asydef}
		usepackage("amsmath");
		usepackage("mathtools");
		usepackage("mathdesign", "adobe-garamond");
		usepackage("bm");
		usepackage("xeCJK");
			texpreamble("\setCJKmainfont[BoldFont = 华文中宋, ItalicFont = 华文楷体]{华文宋体}");
			texpreamble("\setmainfont{AGaramondPro-Regular.otf}");
		usepackage("siunitx");
			texpreamble("\sisetup{
				number-math-rm = \ensuremath,
				inter-unit-product = \ensuremath{{}\cdot{}},
				group-digits = integer,
				group-minimum-digits = 4,
				group-separator = \text{~}
			}");
		texpreamble("%大写正体字母
\newcommand{\rmY}{\mathrm{Y}}

%大写手写体字母
\newcommand{\scF}{\mathcal{F}}
\newcommand{\scH}{\mathcal{H}}
\newcommand{\scI}{\mathcal{I}}
\newcommand{\scJ}{\mathcal{J}}
\newcommand{\scL}{\mathcal{L}}
\newcommand{\scP}{\mathcal{P}}
\newcommand{\scR}{\mathcal{R}}
\newcommand{\scS}{\mathcal{S}}

%小写希腊字母
\renewcommand{\a}{\alpha}
\renewcommand{\b}{\beta}
\newcommand{\g}{\gamma}
\renewcommand{\d}{\delta}
\newcommand{\e}{\epsilon}
\newcommand{\ve}{\varepsilon}
\newcommand{\z}{\zeta}
\newcommand{\h}{\eta}
\newcommand{\q}{\theta}
\renewcommand{\th}{\theta}
\newcommand{\vq}{\vartheta}
\newcommand{\vth}{\vartheta}
\renewcommand{\i}{\iota}
\renewcommand{\k}{\varkappa} %这两个是相反的
\newcommand{\vk}{\kappa} %这两个是相反的
\renewcommand{\l}{\lambda}
\renewcommand{\m}{\mu}
\newcommand{\n}{\nu}
\newcommand{\x}{\xi}
%\renewcommand{\p}{\pi} %\p 已定义为正体π
\newcommand{\vp}{\varpi}
\renewcommand{\r}{\rho}
\newcommand{\vr}{\varrho}
\newcommand{\s}{\sigma}
\newcommand{\vs}{\varsigma}
\renewcommand{\t}{\tau}
\renewcommand{\u}{\upsilon}
\newcommand{\f}{\phi}
\renewcommand{\j}{\varphi}
\newcommand{\vf}{\varphi}
\renewcommand{\c}{\chi}
\newcommand{\y}{\psi}
\renewcommand{\o}{\omega}
\newcommand{\w}{\omega}

%大写希腊字母
\renewcommand{\G}{\varGamma}
\newcommand{\D}{\varDelta}
\newcommand{\Q}{\varTheta}
\newcommand{\Th}{\varTheta}
\renewcommand{\L}{\varLambda}
\newcommand{\X}{\varXi}
\renewcommand{\P}{\varPi}
\renewcommand{\S}{\varSigma}
\renewcommand{\U}{\varUpsilon}
\newcommand{\F}{\varPhi}
\newcommand{\Y}{\varPsi}
\renewcommand{\O}{\varOmega}
\newcommand{\W}{\varOmega}

%常数
\newcommand{\p}{\piup} %圆周率
\newcommand{\ee}{\mathrm{e}} %自然对数基
\newcommand{\ii}{\mathrm{i}} %虚数单位
\newcommand{\hb}{\hbar} %约化Planck常数
\newcommand{\kB}{k_{\mathrm{B}}} %Boltzmann常数
\newcommand{\pfs}{\e_0} %真空介电常数 Permittivity of Free Space

%数学修饰符号
\newcommand{\vecb}[1]{\bm{#1}} %矢量用斜体,根据ISO 80000-2/2009
\renewcommand{\bar}[1]{\overline{#1}} %上横线
\newcommand{\tl}[1]{\widetilde{#1}} %弯号

%数学函数
\newcommand{\abs}[1]{\left\lvert #1 \right\rvert} %绝对值

%算子、算符
\newcommand{\incr}{\Delta}
\newcommand{\dd}{\mathrm{d}} %微分 Differential D
\def\db{{\mathchar'26\mkern-11mu \dd}} %与过程有关的微分 Inexact Differential D
\newcommand{\pd}{\partial} %偏微分 Partial D
\newcommand{\vd}{\deltaup} %变分 Variational D
\newcommand{\laplac}{\nabla^2} %Laplace算子

%复合数学符号
\newcommand{\infint}{\int_{-\infty}^{+\infty}} %上下限无穷的积分
\newcommand{\iTonSum}{\sum_{i = 1}^{n}} %求和 i=1..n
\newcommand{\iToNSum}{\sum_{i = 1}^{N}} %求和 i=1..N
\newcommand{\nEnum}[2]{#1 = 1, \, 2, \, \dots, \, #2} %如x=1,2,...,n
%\newcommand{\coordinateLabel}[2]{#1\text{-}#2} %如p-V坐标

%重定义符号
\newcommand{\approach}{\rightarrow} %趋向于
\newcommand{\eqdef}{\equiv} %定义等号(三线)
\newcommand{\dotTimes}{\cdot} %点乘

%公式用标点、文字
\newcommand{\comma}{\text{,}}
\newcommand{\fullstop}{\text{.}}
\newcommand{\semicomma}{\text{;}}
\newcommand{\const}{\text{const.}}

%正文用符号
\catcode`\。 = \active
\newcommand{。}{.} %句号用句点代替
%\newcommand{\mySec}{{\fontspec{Times New Roman} \S \,}} %TODO:20160302 冲突;章节符号,Garamond 字体不好看");
		texpreamble("\newcommand{\emphA}[1]{{\myHeavy #1}}");
		texpreamble("\newcommand{\emphB}[1]{{\itshape #1}}");
		
		size(6 cm);
		
		defaultpen(fontsize(9 pt));
		
		pen color1 = rgb(0.368417, 0.506779, 0.709798);
		pen color2 = rgb(0.880722, 0.611041, 0.142051);
		pen color3 = rgb(0.560181, 0.691569, 0.194885);
		pen color4 = rgb(0.922526, 0.385626, 0.209179);
		pen color5 = rgb(0.647624, 0.378160, 0.614037);
		pen color6 = rgb(0.772079, 0.431554, 0.102387);
		pen color7 = rgb(0.363898, 0.618501, 0.782349);
		pen color8 = rgb(0.972829, 0.621644, 0.073362);
	\end{asydef}

%浮动体、图标标题
\renewcommand{\thefigure}{\arabic{chapter}.\arabic{figure}}
\usepackage{floatrow}
	\floatsetup[table]{capposition = top}
\usepackage{caption}
	\DeclareCaptionFont{myCationFont}{\CJKfamily{楷体} \small}
	\captionsetup[figure]{
		font = myCationFont,
		labelsep = quad,
		skip = 10 pt,
		position = bottom
	}
	\captionsetup[table]{
		font = myCationFont,
		labelsep = quad,
		skip = 10 pt,
		position = top
	}


%TODO——自定义环境
%自定义定理(定律)
\newenvironment{myThm}[1]
	{\begin{_myThm}[\hskip 2 em #1]}
	{\end{_myThm}}
\newenvironment{myThm*}
	{\begin{_myThm} \hskip 2 em}
	{\end{_myThm}}
%自定义证明(说明)
\newenvironment{myProof}
	{\begin{_myProof} \normalfont \small \CJKfamily{仿宋} \hskip 2 em}
	{\end{_myProof}}
%自定义列表
\newenvironment{myEnum1}
	{\begin{enumerate} [label = \alph*), align = leftalignwithindent, leftmargin = 0 pt, itemsep = 10 pt, parsep = 0 pt, listparindent = 2 em]}
	{\end{enumerate}}
\newenvironment{myEnum2}
	{\begin{enumerate} [label = (\Roman*), align = leftalignwithindent, leftmargin = 2 em, topsep = 0pt, itemsep = 0 pt, parsep = 0 pt, listparindent = 2 em]}
	{\end{enumerate}}
%子公式
\newenvironment{mySubEq}
	{\subequations \renewcommand{\theequation}{\theparentequation-\alph{equation}}}
	{\endsubequations
	\ignorespacesafterend}
%带框公式
\newenvironment{boxedEq}
	{\empheq[box = \fbox]{equation}}
	{\endempheq}
%大括号公式
\newenvironment{braceEq}[1][align]
	{\mySubEq
		\setkeys{EmphEqEnv}{#1}
		\setkeys{EmphEqOpt}{left = \empheqlbrace}
		\EmphEqMainEnv}
	{\endEmphEqMainEnv \endmySubEq}
%自定义表格 {format}{caption}{label}
\newenvironment{myTable}[4][htb]
	{\begin{table}[#1] \centering \small \caption{#3} \label{#4} \begin{tabular}{#2}}
	{\end{tabular} \end{table}}

%TODO——自定义命令
%大写正体字母
\newcommand{\rmY}{\mathrm{Y}}

%大写手写体字母
\newcommand{\scF}{\mathcal{F}}
\newcommand{\scH}{\mathcal{H}}
\newcommand{\scI}{\mathcal{I}}
\newcommand{\scJ}{\mathcal{J}}
\newcommand{\scL}{\mathcal{L}}
\newcommand{\scP}{\mathcal{P}}
\newcommand{\scR}{\mathcal{R}}
\newcommand{\scS}{\mathcal{S}}

%小写希腊字母
\renewcommand{\a}{\alpha}
\renewcommand{\b}{\beta}
\newcommand{\g}{\gamma}
\renewcommand{\d}{\delta}
\newcommand{\e}{\epsilon}
\newcommand{\ve}{\varepsilon}
\newcommand{\z}{\zeta}
\newcommand{\h}{\eta}
\newcommand{\q}{\theta}
\renewcommand{\th}{\theta}
\newcommand{\vq}{\vartheta}
\newcommand{\vth}{\vartheta}
\renewcommand{\i}{\iota}
\renewcommand{\k}{\varkappa} %这两个是相反的
\newcommand{\vk}{\kappa} %这两个是相反的
\renewcommand{\l}{\lambda}
\renewcommand{\m}{\mu}
\newcommand{\n}{\nu}
\newcommand{\x}{\xi}
%\renewcommand{\p}{\pi} %\p 已定义为正体π
\newcommand{\vp}{\varpi}
\renewcommand{\r}{\rho}
\newcommand{\vr}{\varrho}
\newcommand{\s}{\sigma}
\newcommand{\vs}{\varsigma}
\renewcommand{\t}{\tau}
\renewcommand{\u}{\upsilon}
\newcommand{\f}{\phi}
\renewcommand{\j}{\varphi}
\newcommand{\vf}{\varphi}
\renewcommand{\c}{\chi}
\newcommand{\y}{\psi}
\renewcommand{\o}{\omega}
\newcommand{\w}{\omega}

%大写希腊字母
\renewcommand{\G}{\varGamma}
\newcommand{\D}{\varDelta}
\newcommand{\Q}{\varTheta}
\newcommand{\Th}{\varTheta}
\renewcommand{\L}{\varLambda}
\newcommand{\X}{\varXi}
\renewcommand{\P}{\varPi}
\renewcommand{\S}{\varSigma}
\renewcommand{\U}{\varUpsilon}
\newcommand{\F}{\varPhi}
\newcommand{\Y}{\varPsi}
\renewcommand{\O}{\varOmega}
\newcommand{\W}{\varOmega}

%常数
\newcommand{\p}{\piup} %圆周率
\newcommand{\ee}{\mathrm{e}} %自然对数基
\newcommand{\ii}{\mathrm{i}} %虚数单位
\newcommand{\hb}{\hbar} %约化Planck常数
\newcommand{\kB}{k_{\mathrm{B}}} %Boltzmann常数
\newcommand{\pfs}{\e_0} %真空介电常数 Permittivity of Free Space

%数学修饰符号
\newcommand{\vecb}[1]{\bm{#1}} %矢量用斜体,根据ISO 80000-2/2009
\renewcommand{\bar}[1]{\overline{#1}} %上横线
\newcommand{\tl}[1]{\widetilde{#1}} %弯号

%数学函数
\newcommand{\abs}[1]{\left\lvert #1 \right\rvert} %绝对值

%算子、算符
\newcommand{\incr}{\Delta}
\newcommand{\dd}{\mathrm{d}} %微分 Differential D
\def\db{{\mathchar'26\mkern-11mu \dd}} %与过程有关的微分 Inexact Differential D
\newcommand{\pd}{\partial} %偏微分 Partial D
\newcommand{\vd}{\deltaup} %变分 Variational D
\newcommand{\laplac}{\nabla^2} %Laplace算子

%复合数学符号
\newcommand{\infint}{\int_{-\infty}^{+\infty}} %上下限无穷的积分
\newcommand{\iTonSum}{\sum_{i = 1}^{n}} %求和 i=1..n
\newcommand{\iToNSum}{\sum_{i = 1}^{N}} %求和 i=1..N
\newcommand{\nEnum}[2]{#1 = 1, \, 2, \, \dots, \, #2} %如x=1,2,...,n
%\newcommand{\coordinateLabel}[2]{#1\text{-}#2} %如p-V坐标

%重定义符号
\newcommand{\approach}{\rightarrow} %趋向于
\newcommand{\eqdef}{\equiv} %定义等号(三线)
\newcommand{\dotTimes}{\cdot} %点乘

%公式用标点、文字
\newcommand{\comma}{\text{,}}
\newcommand{\fullstop}{\text{.}}
\newcommand{\semicomma}{\text{;}}
\newcommand{\const}{\text{const.}}

%正文用符号
\catcode`\。 = \active
\newcommand{。}{.} %句号用句点代替
%\newcommand{\mySec}{{\fontspec{Times New Roman} \S \,}} %TODO:20160302 冲突;章节符号,Garamond 字体不好看

%强调
\newcommand{\emphA}[1]{{\myHeavy #1}}
\newcommand{\emphB}[1]{{\itshape #1}}

%引入文件
\let \oldInclude = \include
\renewcommand{\include}[1]{{\let \clearpage = \relax \oldInclude{#1}}}
%空行
\newcommand{\blankline}{\mbox{}}
%自定义列表项目(不可删除换行!)
\newcommand{\myItem}[1]{
	\item
	{\bfseries #1}
	
	
}
%引用格式
\newcommand{\secref}[1]{\secSymbol \ref{#1} }
\newcommand{\subsecref}[1]{第\ref{#1}小节}
\newcommand{\egref}[1]{例 \ref{#1}}
%公式标注
\newcommand{\myTag}[1]{\tag*{\CJKfamily{楷体} [#1]}}
\newcommand{\myTagNumbering}[1]{\refstepcounter{equation} \tag*{\CJKfamily{楷体} [#1] \, (\theequation)} }

\title{
	\vspace{-4 cm}
	\bfseries
	热力学与统计物理I
}
\author{
	\CJKfamily{楷体}
	复旦大学\phantom{空格}陈焱
}
\date{
	\CJKfamily{楷体}
	\today
}

\begin{document}
%	\maketitle
%	
%	\frontmatter
%	{\let \cleardoublepage = \clearpage
%		\chapter{绪论}
%			本课程包含两部分内容:热力学、统计物理。

\emphA{热力学(thermodynamics)}是一种自上而下(top-down)的研究方式,它是形而上的、唯象的。用一句话可以形容热力学:“知其然而不知其所以然”。它主要研究宏观物理量之间的关系。

对热力学有重大贡献的物理学家有Carnot、Joule、Clausius、Kelvin等。

\blankline

\emphA{统计物理(statistical Physics)}是一种自下而上(bottom-up)的研究方式,它是形而下的、微观的。

对统计物理有重大贡献的物理学家见表 \ref{TAB_PHYSICIST_IN_STATISTICAL_PHYSICS}。

\begin{table}[htb]
	\begin{tabular}{c|c}
		\hline
		\emphA{阶段} & \emphA{人物} \\
		\hline
		经典统计 & Maxwell、Boltzmann、Gibbs、Einstein等 \\
		量子概念 & Planck、Einstein、Fermi、Dirac、Pauli、Bose等 \\
		量子统计 & von Neumann、Landau、Kramers、Pauli等 \\
		\hline
	\end{tabular}
	\caption{对统计物理有重大贡献的物理学家}
	\label{TAB_PHYSICIST_IN_STATISTICAL_PHYSICS}
\end{table}
统计物理可分为两个阶段。1860年至1902年,人们主要研究近独立子体系(简单地说,就是理想气体);1902年以后,出现系综理论,开始研究凝聚态系统。

统计物理的基础可以概括为\emphB{等概率原理}。
%			
%		\chapter{符号表}
%			\include{Chapters/Front_Symbols}
%			
%		\chapter{缩略词表}
%			\begin{table}[h]
	\begin{tabularx}{\textwidth}{p{8 em}XX}
		ext			&	external			&	外界			\\
		max			&	maximum				&	最大值			\\
		min			&	minimum				&	最小值			\\
	\end{tabularx}
\end{table}

%	}
%	\mainmatter
%	\part{热力学}
		\chapter{热力学基础}
			\section{平衡态及其描述} \label{SEC_平衡态及其描述}
	\subsection{热力学系统}
		热力学研究的对象是\emphA{热力学系统}(简称系统)。它是宏观体系,粒子数的量级至少约 $10^{20}$。与系统相对应的是\emphA{外界},也称为环境或热库。按照与外界的关系,可将系统分为三种:\emphB{孤立系}、\emphB{封闭系}和\emphB{开放系},见表~\ref{TAB_DEFINITION_OF_SYSTEMS}。
		
		\begin{myTable}{cccc}{三种热力学体系}{TAB_DEFINITION_OF_SYSTEMS}
			\toprule
			& \emphA{孤立系} & \emphA{封闭系} & \emphA{开放系} \\%HACK:20160330 表格首行加粗
			\midrule
			物质交换 & \xmark & \xmark & \cmark \\
			能量交换 & \xmark & \cmark & \cmark \\
			\bottomrule
		\end{myTable}
		
	\subsection{平衡态}
		传统热力学研究\emphA{平衡态}。它有两个要素:状态不随时间变化、孤立系。若非孤立系,则称\emphB{稳恒状态},如日光灯。
		
		平衡态只是宏观性质不随时间变化,而微观态仍有变化(微观粒子不断变化)。因此称为\emphA{动态平衡}(就微观状态而言,也称为\emphB{细致平衡})。
		
	\subsection{平衡态的描述} \label{SUBSEC_平衡态的描述}
		平衡态可以利用\emphA{状态变量}(可以测量)来描述,如 $p$、$V$、$T$ 等。需注意,这些量未必互相独立。因此需要选取\emphB{独立}的\emphA{状态参量}。
		
		状态变量可分为两种:\emphA{强度量},如 $p$、$T$、$\r$ 等,它们不随粒子数的增加而增加;\emphA{广延量},如 $V$、$U$、$S$ 等,它们随粒子数的增加而增加。
		
		当某一状态变量可以用其他状态参量来描述时,则称其为一个\emphA{状态函数}。
		
\section{温度;状态方程}
	\subsection{热平衡定律}
		\begin{myThm}{热平衡定律(热力学第零定律)}
			若物体 $A$ 分别与物体 $B$ 和 $C$ 处于热平衡,那么如果让 $B$ 与 $C$ 热接触,它们一定也处于热平衡。
		\end{myThm}
		
		该定律是温度测量的基础:互为热平衡的物体存在一个属于其固有属性的物理量,即\emphA{温度}。一切互为热平衡的物体温度相等。
		
		具体确定温度,需要选定\emphA{温标}。除了常见的摄氏、华氏温标,还有热膨胀温标、热电阻温标、理想气体温标和热力学温标等。
		
	\subsection{物态方程}
		温度与其他状态参量\footnote{
			即独立的状态变量。
		}的函数关系称为\emphA{物态方程}:
		\begin{equation}
			T = f(p, V)
		\end{equation}
		或
		\begin{equation}
			g(p, V, T) = 0 \fullstop
		\end{equation}
		
		Boyle和Mariotte分别于1662年和1676年各自确立了\emphA{Boyle–Mariotte定律}:
		\begin{equation}
			p V = p_0 V_0, \quad \text{若 $T = \const \semicomma$}
		\end{equation}
		1802年,Gay-Lussac确立了\emphA{Gay-Lussac定律}\footnote{
			事实上,在1787年,Charles就已经发现了这一定律,只是当时未发表,也未被人注意。
		}:
		\begin{equation}
			\frac{V}{T} = \frac{V_0}{T_0}, \quad \text{若 $p = \const \fullstop$}
		\end{equation}
		综合两式,可得
		\begin{equation} \label{EQ_IDEAL_GAS_LAW_A}
			\frac{p V}{T} = \frac{p_0 V_0}{T_0} = \text{``const.''} = n R \comma \footnote{
				该式中常数的值显然与物质的量 $n$ 有关,故用带引号的 $\text{``const.''}$ 表示。
			}
		\end{equation}
		即\emphA{理想气体物态方程}
		\begin{boxedEq} \label{EQ_IDEAL_GAS_LAW_B}
			p V = n R T = N \kB T
		\end{boxedEq}
		
		\begin{myProof}
			式~\eqref{EQ_IDEAL_GAS_LAW_A} 推导如下。设某气体初状态为 $p_0$、$V_0$、$T_0$,末状态为 $p$、$V$、$T$。
			
			先保持温度 $T_0$ 不变,则有
			\begin{equation} \label{EQ_PROOF_OF_IDEAL_GAS_LAW_1}
				p_0 V_0 = p V' \semicomma
			\end{equation}
			再假设压强 $p$ 不变,于是
			\begin{equation}
				\frac{V'}{T_0} = \frac{V}{T} \comma
			\end{equation}
			即
			\begin{equation}
				V' = \frac{V T_0}{T} \fullstop
			\end{equation}
			代入 \eqref{EQ_PROOF_OF_IDEAL_GAS_LAW_1}~式,即有
			\begin{equation}
				\frac{p V}{T} = \frac{p_0 V_0}{T_0} = \text{``const.''} \fullstop
			\end{equation}
		\end{myProof}
		
		\blankline
		
		除了通过统计物理推导,也可通过测量\emphB{膨胀系数}、\emphB{压强系数}和\emphB{等温压缩系数}(亦可简称\emphB{压缩系数})来得到物态方程。
		
		膨胀系数 $\a$ 定义为
		\begin{equation} \label{EQ_DEFINITION_OF_EXPANSION_COEFFICIENT}
			\a \eqdef \frac{1}{V} \left( \frac{\pd V}{\pd T} \right)_p \comma
		\end{equation}
		压强系数 $\b$ 定义为
		\begin{equation}
			\b \eqdef \frac{1}{p} \left( \frac{\pd p}{\pd T} \right)_V \comma
		\end{equation}
		等温压缩系数 $\k_T$ 定义为
		\begin{equation}
			\k_T \eqdef -\frac{1}{V} \left( \frac{\pd V}{\pd p} \right)_T \fullstop
		\end{equation}
		
		可以证明,以上三个常数满足下面的关系:
		\begin{equation} \label{EQ_RELATION_OF_ALPHA_BETA_KAPPA}
			\a = \b \k_T p \fullstop
		\end{equation}
		该式说明三者并非独立。通常会直接测量 $\a$ 和 $\k_T$,而通过计算得到 $\b$。
		
		\begin{myProof}
			先证明两个结论。若 $x, \, y, \, z$ 满足 $F(x, y, z) = \const$,其中 $F(x, y, z)$ 是一个可微函数。那么
			\begin{mySubEq}
				\begin{align}
					\dd x &= \left( \frac{\pd x}{\pd y} \right)_z \dd y + \left( \frac{\pd x}{\pd z} \right)_y \dd z \comma \\
					\dd z &= \left( \frac{\pd z}{\pd x} \right)_y \dd x + \left( \frac{\pd z}{\pd y} \right)_x \dd y \fullstop
				\end{align}
			\end{mySubEq}
			把第一式代入第二式,得
			\begin{equation}
				\begin{aligned}
					\dd z &= \left( \frac{\pd z}{\pd x} \right)_y \left[ \left( \frac{\pd x}{\pd y} \right)_z \dd y + \left( \frac{\pd x}{\pd z} \right)_y \dd z \right] + \left( \frac{\pd z}{\pd y} \right)_x \dd y \\
					&= \left[ \left( \frac{\pd z}{\pd x} \right)_y \left( \frac{\pd x}{\pd y} \right)_z + \left( \frac{\pd z}{\pd y} \right)_x \right] \dd y
					+ \left( \frac{\pd z}{\pd x} \right)_y \left( \frac{\pd x}{\pd z} \right)_y \dd z \fullstop
				\end{aligned}
			\end{equation}
			于是
			\begin{equation} \label{EQ_***_dz_EQUAL_***_dy}
				 \left[ 1 - \left( \frac{\pd z}{\pd x} \right)_y \left( \frac{\pd x}{\pd z} \right)_y \right] \dd z
				= \left[ \left( \frac{\pd z}{\pd x} \right)_y \left( \frac{\pd x}{\pd y} \right)_z + \left( \frac{\pd z}{\pd y} \right)_x \right] \dd y \fullstop
			\end{equation}
			因为 $y$ 与 $z$ 是独立的变量,所以 $\dd y$ 与 $\dd z$ 也是独立的。这就要求上式左右两边括号内的项均为零。由左边括号内的项,可得到\emphA{倒数关系}:
			\begin{equation}
				\left( \frac{\pd z}{\pd x} \right)_y \left( \frac{\pd x}{\pd z} \right)_y = 1 \fullstop
			\end{equation}
			类似地,还有
			\begin{equation} \label{EQ_RECIPROCITY_RELATION}
				\left( \frac{\pd z}{\pd y} \right)_x \left( \frac{\pd y}{\pd z} \right)_x = 1 \fullstop
			\end{equation}
			根据 \eqref{EQ_***_dz_EQUAL_***_dy}~式的右侧,有
			\begin{equation} \label{EQ_CYCLIC_RELATION}
				\left( \frac{\pd z}{\pd x} \right)_y \left( \frac{\pd x}{\pd y} \right)_z = -\left( \frac{\pd z}{\pd y} \right)_x \fullstop
			\end{equation}
			代入式~\eqref{EQ_RECIPROCITY_RELATION},便得到\emphA{偏导数三乘积法则}:
			\begin{equation}
				\left( \frac{\pd x}{\pd y} \right)_z \left( \frac{\pd y}{\pd z} \right)_x \left( \frac{\pd z}{\pd x} \right)_y = -1 \fullstop
			\end{equation}
			
			\blankline
			
			然后我们利用式~\eqref{EQ_CYCLIC_RELATION},用热力学中的变量将其写成
			\begin{equation}
				\left( \frac{\pd V}{\pd T} \right)_p = -\left( \frac{\pd V}{\pd p} \right)_T \left( \frac{\pd p}{\pd T} \right)_V \fullstop
			\end{equation}
			代入前文中的定义,就得到了 \eqref{EQ_RELATION_OF_ALPHA_BETA_KAPPA}~式:
			\begin{equation}
				\a = \b \k_T p \fullstop
			\end{equation}
		\end{myProof} 
		
		\blankline
		
		考虑分子之间的相互作用后,对理想气体进行修正,这就是\emphA{van der Waals气体}。其物态方程为
		\begin{equation} \label{EQ_VAN_DER_WAALS_GAS_STATE_EQUATION}
			\left( p + \frac{n^2 a}{V^2} \right) (V - n b) = n R T \comma
		\end{equation}
		其中 $n^2 a / V^2$ 代表分子之间吸引力所引起的修正,而 $n b$ 则代表排斥力所引起的修正。
		
\section{功;热力学第一定律}
	\subsection{准静态过程的功}
		静态过程即平衡态。
		
		\emphA{准静态过程}指过程中的每一步都是平衡态。这就要求外界的条件变化地足够缓慢。令 $\t$ 为外界条件变化的特征时间,$\incr t$ 为系统趋于与外界条件对应的平衡态的特征时间(即\emphA{弛豫时间}),那么准静态过程相当于
		\begin{equation}
			\frac{\t}{\incr t} \approach 0
		\end{equation}
		的极限。
		
		一个过程中,若每一步都可以在相反的方向进行而不引起外界的变化,则称为\emphA{可逆过程}。可逆过程的实质是没有耗散。
		
		下面举两个准静态过程的例子。
		
		\begin{myEnum1}
			\myItem{$p V$ 系统}
				考虑盛于带活塞容器内的气体。气体体积变化 $\dd V$ 时,\emphB{外界对系统}做功
				\begin{equation}
					\db W = -p \dd V \fullstop
				\end{equation}
				这里的 $p$ 指外界对系统的压强。由于是准静态过程,它又等于容器内气体对器壁的压强。外界压力作用下体积减小,$\dd V < 0$;而外界却对系统做正功,因此会出现负号。如果令 $W'$ 为\emphB{系统对外界}做的功,则有
				\begin{equation}
					\db W' = - \dd W = p \dd V \fullstop
				\end{equation}
				
				体积由 $V_1$ 变化到 $V_2$,外界对系统做的总功
				\begin{equation}
					W = -\int_{V_1}^{V_2} p \dd V \fullstop
				\end{equation}
				
				\begin{figure}[h]
					\centering
					
					\begin{asy}
						pair O = (0, 0), x_axes = (10, 0), y_axes = (0, 10);
						
						pair p1 = (2.2, 7.5), p2 = (8, 2.5), p3 = (4, 4.5);
						pair p1_x = (p1.x, 0), p2_x = (p2.x, 0);
						
						fill(p1_x--p1..p3..p2--p2_x--cycle, color1 + opacity(0.2));
						
						draw(Label("$V$", EndPoint), O--x_axes, Arrow);
						draw(Label("$p$", EndPoint), O--y_axes, Arrow);
						
						draw(p1..p3..p2, linewidth(1) + color1);
						draw(p1_x--p1, dashed + color1);
						draw(p2_x--p2, dashed + color1);
						
						label("$O$", O, SW);
						label("状态 $1$", p1, (0, 2));
						label("状态 $2$", p2, (0.5, 2));
						label("$V_1$", p1_x, S);
						label("$V_2$", p2_x, S);
					\end{asy}
					\caption{$p \text{-} V$ 系统的状态空间}
					\label{FIG_pV_DIAGRAM}
				\end{figure}
				
				在 $p \text{-} V$ 图中标出状态1和状态2,则连接它们的曲线就表示一个准静态过程,曲线下的面积就等于 $-W$,如图~\ref{FIG_pV_DIAGRAM} 所示。显然,$W$ 与路径有关,即不是一个全微分,因此用“$\db W$”表示微功。
				
			\myItem{电介质极化系统}
				
				考虑均匀电场 $\vecb{E}$ 中的均匀电介质。电位移 $\vecb{D}$ 变化 $\dd \vecb{D}$ 时,电场做功
				\begin{equation}
					\db W = V \vecb{E} \dotTimes \dd \vecb{D} \comma
				\end{equation}
				其中的 $V$ 表示电介质的体积。这里,$\vecb{E}$ 是变化的外因,$\vecb{D}$ 则是内果。令 $\vecb{P}$ 为极化强度,则
				\begin{equation}
					\vecb{D} = \pfs \vecb{E} + \vecb{P} \fullstop
				\end{equation}
				于是
				\begin{equation}
					\begin{aligned}
						\db W &= V \pfs \vecb{E} \dotTimes \dd \vecb{E} + V \vecb{E} \dotTimes \dd \vecb{P} \\
						&= V \dd \left( \frac{1}{2} \pfs \vecb{E}^2 \right) + V 
						\vecb{E} \dotTimes \dd \vecb{P} \fullstop
					\end{aligned}
				\end{equation}
				式中的第一项代表电场能量的变化,第二项则代表\emphB{极化功}。
		\end{myEnum1}
		
		对以上两种情况进行推广,可得
		\begin{align} \label{EQ_GENERAL_WORK}
			\db W &= Y_1 \dd y_1 + Y_2 \dd y_2 + \cdots Y_n \dd y_n \notag \\
			&= \iTonSum Y_i \dd y_i \fullstop
		\end{align}
		这里的 $Y_i$ 是\emphA{广义力},如压强 $p$、电场强度 $\vecb{E}$、磁场强度 $\vecb{H}$ 等;$y_i$ 是\emphA{广义坐标},如体积 $V$、电位移 $\vecb{D}$、磁感应强度 $\vecb{B}$ 等。
		
	\subsection{非静态过程}
		\begin{myEnum1}
			\myItem{等容过程}
				由于体积不变,$\dd V = 0$,因此
				\begin{equation} \label{EQ_WORK_IN_ISOCHORIC_PROCESS}
					\db W = 0 \fullstop
				\end{equation}
				
			\myItem{等压过程}
				\emphB{外界}压强不变,即 $p_{\text{ext}} = \const$,因此体积从 $V_1$ 变化到 $V_2$ 时外界对系统做功
				\begin{equation}
					W = -p_{\text{ext}} (V_2 - V_1) = -p_{\text{ext}} \incr V \fullstop
				\end{equation}
				若系统初、终态压强相等都处于平衡态,有
				\begin{equation}
					p_1 = p_2 = p_{\text{ext}} \eqdef p \fullstop
				\end{equation}
				于是功
				\begin{equation} \label{EQ_WORK_IN_ISOBARIC_PROCESS}
					W = -p (V_2 - V_1) = -p \incr V \fullstop
				\end{equation}
		\end{myEnum1}
		
	\subsection{热力学第一定律}
		\emphA{热力学第一定律}的主要建立者有Mayer、Joule、Helmholtz、Carnot等。它描述了\emphB{功}、\emphB{热量}、\emphB{内能}三者之间的关系,是\emphA{能量守恒定律}在宏观热现象过程中的表现形式。
		\begin{myThm}{能量守恒定律}
			自然界中的一切物质都具有能量。能量有各种不同的形式,能够从一种形式转化为另一种形式,从一个物体传递给另一个物体。在转化和传递中,能量的总量不变。
		\end{myThm}
		
		\begin{myEnum1}
			\myItem{绝热过程}
				从状态1变化到状态2的过程中,只有外界做功,因此
				\begin{equation}
					U_2 - U_1 = W_\text{a} \comma
				\end{equation}
				其中的 $W_\text{a}$ 表示绝热功。
				
			\myItem{非绝热过程}
			由于既有外界做功,又有热量传递,因此
			\begin{equation} \label{EQ_1ST_LAW_IN_INTEGRAL_FORM}
				U_2 - U_1 = W + Q \fullstop
			\end{equation}
			写成微分形式,为
			\begin{boxedEq} \label{EQ_1ST_LAW_IN_DIFFERENTIAL_FORM}
				\dd U = \db W + \db Q
			\end{boxedEq}
		\end{myEnum1}
		这实际上就是热力学第一定律最常用的表述。
		
		设两个全同系统内能分别为 $U_1$、$U_2$。则总内能 $U_\text{total}$ 除了 $U_1 + U_2$,还需包括由于界面效应所导致的 $U_{12}$。但在热力学中,界面效应基本可以忽略,因此近似有
		\begin{equation}
			U_\text{total} = U_1 + U_2 \fullstop
		\end{equation}
		这就是说,内能也是一个广延量。
		
\section{热容与焓;理想气体的性质} \label{SEC_热容与焓;理想气体的性质}
	\subsection{热容与焓} \label{SUBSEC_热容与焓}
		对于过程 $y$,定义\emphA{热容(heat capacity)}
		\begin{equation}
			C_y \eqdef \frac{\db Q_y}{\dd T} \comma
		\end{equation}
		式中的 $\db Q_y$ 是温度升高 $\dd T$ 时系统所吸收的热量。
		
		对于\emphB{等容过程},有\emphA{定容热容}
		\begin{equation}
			C_V \eqdef \frac{\db Q_V}{\dd T} \fullstop
		\end{equation}
		根据式~\eqref{EQ_WORK_IN_ISOCHORIC_PROCESS},可知
		\begin{equation}
			\db Q_V = \dd U - \db W = \dd U - 0 = \dd U \comma
		\end{equation}
		因此
		\begin{equation} \label{EQ_HEAT_CAPACITY_IN_CONST_V}
			C_V = \left( \frac{\pd U}{\pd T} \right)_V \fullstop
		\end{equation}
		
		对于\emphB{等压过程},有\emphA{定压热容}
		\begin{equation}
			C_p \eqdef \frac{\db Q_p}{\dd T} \fullstop
		\end{equation}
		根据式~\eqref{EQ_WORK_IN_ISOBARIC_PROCESS},可知
		\begin{equation} \label{EQ_dQ_IN_CONST_p}
			\db Q_p = \dd U - \db W = \dd U + p \dd V \comma
		\end{equation}
		因此
		\begin{equation} \label{EQ_HEAT_CAPACITY_IN_CONST_p}
			C_p = \left( \frac{\pd U}{\pd T} \right)_p + p \left( \frac{\pd V}{\pd T} \right)_p \fullstop
		\end{equation}
		
		可以发现,
		\begin{equation}
			C_p = C_V + p \left( \frac{\pd V}{\pd T} \right)_p \fullstop
		\end{equation}
		这是它们的关系之一。\footnote{
			$\pd U / \pd T$ 的下标,在式~\eqref{EQ_HEAT_CAPACITY_IN_CONST_V} 中是 $V$,而在式~\eqref{EQ_HEAT_CAPACITY_IN_CONST_p} 中则是 $p$。%TODO:20160312 偏微分关系
		}%TODO:20160330 下标的更换是不对的,或者有其他要求
		
		\blankline
		
		引入一个新的物理量——\emphA{焓(enthalpy)},其定义为
		\begin{equation}
			H \eqdef U + p V \fullstop
		\end{equation}
		由于 $U$、$p$ 和 $V$ 均是状态函数,因此焓也是状态函数。利用焓,可将式~\eqref{EQ_HEAT_CAPACITY_IN_CONST_p} 改写为
		\begin{equation} \label{EQ_HEAT_CAPACITY_IN_CONST_p_WITH_H}
			C_p = \left( \frac{\pd H}{\pd T} \right)_p \comma
		\end{equation}
		同时将式~\eqref{EQ_dQ_IN_CONST_p} 改写为
		\begin{equation}
			\db Q_p = \dd H \fullstop
		\end{equation}
		这就是说,\emphB{在等压过程中,物体吸收的热量等于焓的增加量}。
		
		很明显,热容与焓均是广延量。单位质量的热容称为\emphA{比热容(specific heat capacity)}或\emphA{比热(specific heat)},它是强度量。
		
	\subsection{理想气体的性质} \label{SUBSEC_理想气体的性质}
		\begin{figure}[h]
			\begin{asy}
				pair p1 = (0, 7), p2 = (0, 0), p3 = (12, 0), p4 = (p3.x, p1.y), p5 = (0, 6), p6 = (p3.x, p5.y);
				pair m1 = (p1+p4)/2, m2 = (p2+p3)/2;
				transform myReflect = reflect(m1, m2);
				
				real boxWidth = 3, boxHeight = 2;
				pair pA1 = (1.8, 2.5), pA2 = (pA1.x+boxWidth, pA1.y), pA3 = (pA2.x, pA1.y+boxHeight), pA4 = (pA1.x, pA3.y);
				path boxA = pA1--pA2--pA3--pA4--cycle;
				path boxB = myReflect*boxA;
				
				real pipeSize = 0.2, pipeHeight = 3.5;
				pair ppA1 = (pA4.x+(boxWidth-pipeSize)/2, pA4.y), ppA2 = (ppA1.x, ppA1.y+pipeHeight), ppA3 = (ppA1.x+pipeSize, ppA1.y), ppA4 = (ppA3.x, ppA2.y-pipeSize);
				pair ppB1 = myReflect*ppA1, ppB2 = myReflect*ppA2, ppB3 = myReflect*ppA3, ppB4 = myReflect*ppA4;
				path pipe = ppA1--ppA2--ppB2--ppB1--ppB3--ppB4--ppA4--ppA3--cycle;
				
				real valveHeight = 0.6;
				pair pv1 = (m1.x, ppA2.y-(valveHeight+pipeSize)/2), pv2 = (pv1.x, pv1.y+valveHeight);
				
				real thermometerHeight = 5, thermometerR = 0.2;
				pair pt1 = (1, 3), pt2 = (pt1.x, pt1.y+thermometerHeight);
				
				fill(p5--p2--p3--p6--cycle, color1+opacity(0.2));
				draw(p1--p2--p3--p4, linewidth(1)+color1);
				draw(boxA, linewidth(1)+color2);
				draw(boxB, linewidth(1)+color2);
				draw(pipe, linewidth(1)+color2);
				draw(pv1--pv2, linewidth(2)+color3);
				draw(pt1--pt2, linewidth(2)+color4);
				fill(circle(pt1, thermometerR), color4);
				
				label("A", (ppA1+ppA3)/2, (0, -3));
				label("B", (ppB1+ppB3)/2, (0, -3));
				
				pair pwLabel1 = (p3.x-1, p3.y+1.4), pwLabel2 = (p3.x+1, p3.y+2.5);
				draw(pwLabel1--pwLabel2);
				label("水槽", pwLabel2, E);
				
				pair ptLabel1 = (pt1.x-0.25, pt2.y-1), ptLabel2 = (pt1.x-1.5, pt2.y+0.4);
				draw(ptLabel1--ptLabel2);
				label("温度计", ptLabel2, N);
				
				pair pvLabel1 = (pv1.x+0.25, pv2.y-0.1), pvLabel2 = (pv1.x+1.5, pv2.y+1);
				draw(pvLabel1--pvLabel2);
				label("阀门", pvLabel2, N);
			\end{asy}
			\caption{Joule实验的装置}
			\label{FIG_JOULE_EXPERIMENT}
		\end{figure}
		
		先介绍Joule实验(1845年)。设有一容器,分为A、B两个相同的部分。将它们置于水槽内,并通过带阀门的细管相连,水温可以通过温度计测量,如图~\ref{FIG_JOULE_EXPERIMENT} 所示。
		
		首先,在A内充满气体,而使B保持真空。然后打开阀门,气体将\emphB{自由膨胀},并充满整个容器。Joule的实验结果是前后水温不变。
		
		由于是等容过程,因此 $W = 0$;水温不变,说明 $Q = 0$。根据热力学第一定律,有 $\incr U = 0$。假设内能是温度和体积的函数,即
		\begin{equation}
			U = U(T, \, V) \fullstop
		\end{equation}%TODO:20160312 此处推导有问题?
		根据\emphB{偏导数三乘积法则},可知
		\begin{equation}
			\left( \frac{\pd U}{\pd V} \right)_T \left( \frac{\pd V}{\pd T} \right)_U \left( \frac{\pd T}{\pd T} \right)_V = -1 \fullstop
		\end{equation}
		于是
		\begin{equation}
			\left( \frac{\pd U}{\pd V} \right)_T 
			= -\left( \frac{\pd U}{\pd T} \right)_V \left( \frac{\pd T}{\pd V} \right)_U 
			= 0 \fullstop
		\end{equation}
		这就说明 $U = U(T)$,即内能只和温度有关。
		
		不过,Joule实验过于粗糙,更精确的实验表明实际气体的内能不仅与 $T$ 有关,还与 $V$ 有关。但对于理想气体,以上结论仍然成立,因此理想气体具有如下两条性质:
		\begin{mySubEq}
			\begin{empheq}[left=\empheqlbrace]{align}
				& p V = n R T \comma \label{EQ_IDEAL_GAS_PROPERTY_STATE_EQUATION} \\
				& U = U(T) \label{EQ_IDEAL_GAS_PROPERTY_INTERNAL_ENERGY} \fullstop
			\end{empheq}
		\end{mySubEq}
		就目前而言,这两条性质彼此是独立的。但利用热力学第二定律或统计物理,可以由 \eqref{EQ_IDEAL_GAS_PROPERTY_STATE_EQUATION}~式 推导 \eqref{EQ_IDEAL_GAS_PROPERTY_INTERNAL_ENERGY}~式。[见\secref{SEC_Maxwell关系}\subsecref{SUBSEC_简单应用_OF_MAXWELL关系}中的\egref{EG_pd_U/pd_V_WITH_FIXED_T}]%FIXME:20160401 后续交叉引用
		
		对于理想气体,根据焓的定义,可得
		\begin{equation}
			H = U + p V = U(T) + n R T = H(T) \comma
		\end{equation}
		即焓也仅是温度的函数。从而根据式~\eqref{EQ_HEAT_CAPACITY_IN_CONST_V} 和 \eqref{EQ_HEAT_CAPACITY_IN_CONST_p_WITH_H},又有
		\begin{align}
			C_p - C_V 
			&= \left( \frac{\pd H}{\pd T} \right)_p - \left( \frac{\pd U}{\pd T} \right)_V \notag \\
			&= \frac{\dd H}{\dd T} - \frac{\dd U}{\dd T} \notag \\
			&= \frac{\dd \; (p V)}{\dd T} 
			= \frac{\dd \; (n R T)}{\dd T} 
			= n R \fullstop \label{EQ_C_p-C_V_FOR_IDEAL_GAS}
		\end{align}
		
		定义\emphA{热容比(heat capacity ratio)} $\g$ \footnote{
			也称为\emphA{绝热指数(adiabatic index)},原因见\subsecref{SUBSEC_绝热过程的过程方程}。
		}
		为 $C_p$ 与 $C_V$ 之比:
		\begin{equation} \label{EQ_DEF_OF_HEAT_CAPACITY_RATIO}
			\g \eqdef \frac{C_p}{C_V} = \g(T) \fullstop
		\end{equation}
		这里 $\g = \g(T)$ 是显而易见的。
		
		由式~\eqref{EQ_C_p-C_V_FOR_IDEAL_GAS} 和 \eqref{EQ_DEF_OF_HEAT_CAPACITY_RATIO},可以解得
		\begin{mySubEq}
			\begin{empheq}[left=\empheqlbrace]{align}
				& C_V = \frac{1}{\g - 1} n R \comma \label{EQ_C_V_BY_HEAT_CAPACITY_RATIO}\\
				& C_p = \frac{\g}{\g - 1} n R \fullstop \label{EQ_C_p_BY_HEAT_CAPACITY_RATIO}
			\end{empheq}
		\end{mySubEq}
		因为 $\g$ 可以通过实验测量,由此算出热容后,就可以确定理想气体的内能与焓:
		\begin{mySubEq}
			\begin{empheq}[left=\empheqlbrace]{align}
				& U(T) = \int C_V(T) \dd T + U_0 \comma \\
				& H(T) = \int C_p(T) \dd T + H_0 \comma
			\end{empheq}
		\end{mySubEq}
		其中的 $U_0$ 和 $H_0$ 是积分常数。写成微分形式,为
		\begin{mySubEq}
			\begin{empheq}[left=\empheqlbrace]{align}
			& \dd U = C_V(T) \dd T \comma \label{EQ_dU=Cv*dT_IDEAL_GAS} \\
			& \dd H = C_p(T) \dd T \fullstop \label{EQ_dH=Cp*dT_IDEAL_GAS}
			\end{empheq}
		\end{mySubEq}
		
	\subsection{绝热过程的过程方程} \label{SUBSEC_绝热过程的过程方程}
		本节叙述均针对理想气体。
		
		根据理想气体状态方程,有
		\begin{equation}
			p V = n R T \fullstop
		\end{equation}
		于是
		\begin{equation}
			\dd T = \frac{p \dd V + V \dd p}{n R} \fullstop
		\end{equation}
		对于绝热过程,根据热力学第一定律,有
		\begin{align}
			\db Q &= \dd U + p \dd V \notag \\
			&= C_V \dd T + p \dd V \notag \\
			&= C_V \, \frac{p \dd V + V \dd p}{n R} + p \dd V \notag \\
			&= \frac{\g}{\g - 1} p \dd V + \frac{1}{\g - 1} V \dd p \myTag{见式~\eqref{EQ_C_V_BY_HEAT_CAPACITY_RATIO}} \\
			&= 0 \comma
		\end{align}
		因此
		\begin{align}
			&\mathrel{\phantom{\implies}} V \dd p + \g p \dd V = 0 \\
			&\implies \frac{\dd p}{p} + \g \frac{\dd V}{V} = 0 \notag \\
			&\implies \ln p + \g \ln V = \ln \left(p V^{\g} \right) = 0 \notag \\
			&\implies p V^{\g} = \label{EQ_STATE_EQUATION_OF_ADIABATIC_PROCESS_IN_p_V} \const \footnotemark
		\end{align} \footnotetext{这一步推导假定 $\g$ 为常数,因此可以直接积分。}
		改用其他变量,可把该式写成
		\begin{equation} \label{EQ_STATE_EQUATION_OF_ADIABATIC_PROCESS_IN_p_T}
			p^{(1 - \g) / \g} T = \const
		\end{equation}
		或
		\begin{equation} \label{EQ_STATE_EQUATION_OF_ADIABATIC_PROCESS_IN_T_V}
			T V^{\g - 1} = \const
		\end{equation}
		的形式。
		
		\begin{myExample}[海拔与气温的关系]
			下面推导\emphB{气温垂直递减率}。首先考虑\emphB{干燥空气}的温度\emphB{绝热}递减率。假设空气是理想气体。
			
			因为是绝热过程,因此
			\begin{equation}
				p^{(1 - \g) / \g} T = \const
			\end{equation}
			两边求微分,得
			\begin{equation}
				\frac{1 - \g}{\g} p^{\frac{1 - \g}{\g} - 1} T \dd p + p^{\frac{1 - \g}{\g} - 1} \dd T = 0 \comma
			\end{equation}
			即
			\begin{equation} \label{EQ_dT/dp_IN_EXAMPLE_OF_LAPSE_RATE}
				\frac{\dd T}{\dd p} = \frac{1 - \g}{\g} \frac{T}{p} \fullstop
			\end{equation}
			
			假设大气处于平衡状态,则有
			\begin{equation}
				\dd p = -\r g \dd z \comma
			\end{equation}
			其中的 $\r$ 是空气密度,
			\begin{equation}
				\r = \frac{m}{V} = \frac{n M}{n R T / p} = \frac{p M}{R T} \comma
			\end{equation}
			这里的 $M$ 是空气的平均摩尔质量, $g$ 是重力加速度,$m$、$V$ 分别是一定量空气的质量和体积。代入 \eqref{EQ_dT/dp_IN_EXAMPLE_OF_LAPSE_RATE} 式,可得
			\begin{equation}
				\frac{\dd T}{\dd z} = -\frac{1 - \g}{\g} \frac{\r g T}{p} = -\frac{1 - \g}{\g} \frac{M g}{R} \fullstop
			\end{equation}
			代入空气的热容比 $\g = 1.4$、平均摩尔质量 $M = \SI{28.8e-3}{\kg\per\mol}$ 等数值,可得
			\begin{equation}
				\frac{\dd T}{\dd z} = \SI{-9.7}{\kelvin\per\km} \fullstop
			\end{equation}
			
			\blankline
			
			若为水汽饱和的湿空气,有下面的近似公式:%TODO:20160315 参考文献
			\begin{equation} \label{EQ_SATURATED_ADIABATIC_LAPSE_RATE}
				\frac{\dd T}{\dd z}
				= -g \, \frac{1 + \dfrac{L_\text{vap} r}{R_\text{s,\,dry} T}}{c_{p,\,\text{dry}} + \dfrac{L_\text{vap}^2 r}{R_\text{s,\,water} T^2}}
				= -g \, \frac{1 + \dfrac{L_\text{vap} r}{R_\text{s,\,dry} T}}{c_{p,\,\text{dry}} + \dfrac{L_\text{vap}^2 r \e}{R_\text{s,\,dry} T^2}} \comma
			\end{equation}
			式中的各符号见表~\ref{TAB_SYMBOLS_IN_SATURATED_ADIABATIC_LAPSE_RATE}。%FIXME:20160401 qed位置
			
			\begin{myTable}{Mcc}{式~\eqref{EQ_SATURATED_ADIABATIC_LAPSE_RATE} 中所用到的符号}{TAB_SYMBOLS_IN_SATURATED_ADIABATIC_LAPSE_RATE}
				\toprule
				\text{\emphA{符号}} & \emphA{说明} & \emphA{数值} \\%HACK:20160330 表格首行加粗
				\midrule
				g & 重力加速度 & \SI{9.8076}{\metre\per\second\squared} \\
				L_\text{vap} & 水的汽化热 & \SI{2257}{\kilo\joule\per\kg} \\
				c_{p,\,\text{dry}} & 干燥空气的定压比热容 & \SI{1003.5}{\joule\per\kg\per\kelvin} \\
				R_\text{s,\,dry} & 干燥空气的气体常数 & \SI{287}{\joule\per\kg\per\kelvin} \\
				R_\text{s,\,water} & 水蒸气的气体常数 & \SI{461.5}{\joule\per\kg\per\kelvin} \\
				\e = R_\text{s,\,dry} / R_\text{s,\,water} & 干燥空气与水蒸气的气体常数之比 & 0.622 \\
				e & 饱和空气的水蒸气分压 & —— \\
				p & 饱和空气的气压 & —— \\
				r = \e e / (p - e) & 水蒸气的质量与干燥空气质量的混合比例 & —— \\
				T & 饱和空气的温度 & —— \\
				\bottomrule
			\end{myTable}
		\end{myExample}
		
\section{理想气体与Carnot循环;热力学第二定律} \label{SEC_理想气体与Carnot循环_热力学第二定律}
	\subsection{Carnot循环} \label{SUBSEC_Carnot循环}
		\emphA{Carnot循环}分为四个过程,如图~\ref{FIG_CARNOT_CYCLE} 所示。
		\begin{myEnum2}
			\item 等温膨胀:$(T_\text{H}, \, V_1) \rightarrow (T_\text{H}, \, V_2)$,
			\item 绝热膨胀:$(T_\text{H}, \, V_2) \rightarrow (T_\text{C}, \, V_3)$,
			\item 等温压缩:$(T_\text{C}, \, V_3) \rightarrow (T_\text{C}, \, V_4)$,
			\item 绝热压缩:$(T_\text{C}, \, V_4) \rightarrow (T_\text{H}, \, V_1)$,
		\end{myEnum2}
		这里的 $T_\text{H}$ 和 $T_\text{C}$ 分别指高温和低温,并且还有 $V_1 < V_2$,$V_4 < V_3$。
		
		\begin{figure}[h]
			\begin{asy}
				import graph;
				pair O = (0, 0), x_axes = (10, 0), y_axes = (0, 10);
				draw(Label("$V$", EndPoint), O--x_axes, Arrow);
				draw(Label("$p$", EndPoint), O--y_axes, Arrow);
				
				real gamma = 2.5;
				real x1 = 2, x4 = 9;
				real c1 = 10, c2 = 18;
				real c3 = c2*x1^(gamma-1), c4 = c1*x4^(gamma-1);
				real x2 = x1*(c1/c2)^(1/(1-gamma)), x3 = x4*(c2/c1)^(1/(1-gamma));
				
				path path1 = graph(new real(real x) {return c2/x;}, x1, x3);
				path path2 = graph(new real(real x) {return c4/x^gamma;}, x3, x4);
				path path3 = reverse(graph(new real(real x) {return c1/x;}, x2, x4));
				path path4 = reverse(graph(new real(real x) {return c3/x^gamma;}, x1, x2));
				//path path_TH = graph(new real(real x) {return c2/x;}, x3, 8);
				//path path_TC = graph(new real(real x) {return c1/x;}, 1.7, x2);
				
				pair p1 = (x1, c2/x1), p2 = (x3, c2/x3), p3 = (x4, c1/x4), p4 = (x2, c1/x2);
				
				pen pen1 = linewidth(1)+color1;
				
				fill(path1 & path2 & path3 & path4 & cycle, color1+opacity(0.2));
				
				//draw(path_TH, dashed + color1);
				//draw(path_TC, dashed + color1);
				draw(path1, pen1, Arrow(position = Relative(0.7), arrowhead = HookHead, size = 4));
				draw(path2, pen1, Arrow(position = Relative(0.5), arrowhead = HookHead, size = 4));
				draw(path3, pen1, Arrow(position = Relative(0.7), arrowhead = HookHead, size = 4));
				draw(path4, pen1, Arrow(position = Relative(0.45), arrowhead = HookHead, size = 4));
				
				draw(Label("$V_1$", EndPoint, black), p1--(p1.x, 0), dashed + color1);
				draw(Label("$V_2$", EndPoint, black), p2--(p2.x, 0), dashed + color1);
				draw(Label("$V_3$", EndPoint, black), p3--(p3.x, 0), dashed + color1);
				draw(Label("$V_4$", EndPoint, black), p4--(p4.x, 0), dashed + color1);
				
				label("状态1", p1, N);
				label("状态2", p2, NE);
				label("状态3", p3, E);
				label("状态4", p4, SW, Fill(white));
				
				label("I", path1, align = Relative(W));
				label("II", path2, align = Relative(W));
				label("III", path3, align = Relative(W));
				label("IV", path4, align = Relative(W), Fill(white));
			\end{asy}
			\caption{Carnot循环示意图}
			\label{FIG_CARNOT_CYCLE}
		\end{figure}
		
		利用热力学第一定律,有
		\begin{equation}
			\oint \dd U = \oint \db Q + \oint \db W = 0 \comma
		\end{equation}
		其中的“$\oint$”代表沿循环过程的积分。
		
		整个过程中对外做的净功(它等于图~\ref{FIG_CARNOT_CYCLE} 中曲线包围起来的面积)
		\begin{equation}
			W' = -\oint \db W = \oint \db Q = Q_\text{H} + Q_\text{C} \comma
		\end{equation}
		其中的 $Q_\text{H}$ 和 $Q_\text{C}$ 分别为高温和低温时吸收的热量(可以有正负)。
		
		若工作物质为理想气体,则有
		\begin{align}
			Q_\text{H} &= \incr U_\text{I} - W_\text{I} \myTag{热力学第一定律}\\
			&= 0 - W_\text{I} \notag \\
			&= \int_{V_1}^{V_2} p \dd V \notag \\
			&= n R T_\text{H} \int_{V_1}^{V_2} \frac{\dd V}{V} \myTag{根据 $p V = n R T$} \\
			&= n R T_\text{H} \ln \frac{V_2}{V_1}
			> 0 \label{EQ_Q_H_IN_CARNOT_CYCLE} \fullstop
		\end{align}
		同理,还有
		\begin{equation} \label{EQ_Q_C_IN_CARNOT_CYCLE}
			Q_\text{C} = -n R T_\text{C} \ln \frac{V_3}{V_4} < 0 \fullstop
		\end{equation}
		
		根据绝热过程的过程方程~\eqref{EQ_STATE_EQUATION_OF_ADIABATIC_PROCESS_IN_T_V},有
		\begin{mySubEq}
			\begin{empheq}[left=\empheqlbrace]{align}
				& T_\text{H} V_2^{\g - 1} = T_\text{C} V_3^{\g - 1} \comma \\
				& T_\text{H} V_1^{\g - 1} = T_\text{C} V_4^{\g - 1} \comma
		\end{empheq}
		\end{mySubEq}
		两边分别相除,得
		\begin{equation} \label{EQ_V2/V1_IN_CARNOT_CYCLE}
			\frac{V_2}{V_1} = \frac{V_3}{V_4} \fullstop
		\end{equation}
		
		定义\emphA{热机效率}
		\begin{equation}
			\h \eqdef \frac{W'}{Q_\text{H}} 
			= \frac{Q_\text{H} - \abs{Q_\text{C}}}{Q_\text{H}} 
			= 1 - \frac{\abs{Q_\text{C}}}{Q_\text{H}} \fullstop
		\end{equation}
		对于理想气体,代入式~\eqref{EQ_Q_H_IN_CARNOT_CYCLE} 和 \eqref{EQ_Q_C_IN_CARNOT_CYCLE},并利用式~\eqref{EQ_V2/V1_IN_CARNOT_CYCLE},可得
		\begin{align}
			\h &= 1 - \frac{T_\text{C}}{T_\text{H}} \frac{\ln (V_3 / V_4)}{\ln (V_2 / V_1)} \notag \\
			&= 1 - \frac{T_\text{C}}{T_\text{H}} \label{EQ_CARNOT_CYCLE_EFFICIENCY_WITH_IDEAL_GAS} \fullstop
		\end{align}%TODO:20160320 非理想气体的情况,见作业
		
		若Carnot循环反向进行,就成为\emphA{Carnot制冷机}。其\emphB{制冷效率}定义为
		\begin{equation}
			\ve = \frac{Q_\text{C}}{W}
			= \frac{Q_\text{C}}{Q_\text{H} - Q_\text{C}}
			= \frac{T_\text{C}}{T_\text{H} - T_\text{C}} \comma
		\end{equation}
		它通常是大于 $1$ 的。
		
	\subsection{热力学第二定律}
		\emphA{热力学第二定律}解决了有关过程\emphB{方向性}的问题,它的主要建立者有Carnot、Clausius、Kelvin等。
		
		\begin{myThm}{热力学第二定律(Kelvin表述)}
			不可能从单一热源吸热使之完全变为有用的功而不产生其他影响,即第二类永动机不可能实现。
		\end{myThm}
		\begin{myThm}{热力学第二定律(Clausius表述)}
			不可能把热量从低温物体传到高温物体而不产生其他影响。
		\end{myThm}
		
		\begin{myProof}%TODO:20160318 图片
			Kelvin表述 $\implies$ Clausius表述:
			
			采用反证法,即证明 $\neg \, (\text{Clausius表述}) \implies \neg \, (\text{Kelvin表述})$。
			
			Carnot热机A工作于高温热源 $T_\text{H}$ 和低温热源 $T_\text{C}$ 之间。它从 $T_\text{H}$ 处吸收热量 $Q_\text{H}$,向 $T_\text{C}$ 放出热量 $Q_\text{C}$,并做功 $W = Q_\text{H} - Q_\text{C}$。假设Clausius表述不成立,就可以在不产生其他影响的前提下,使低温热源获得的热量 $Q_\text{C}$ 重新回到高温热源。净结果便是从单一热源 $T_\text{H}$ 吸收了热量 $Q_\text{H} - Q_\text{C}$,并将其完全转化为功,这就违背了Kelvin表述。因此原假设不成立,即有 $\neg \, (\text{Clausius表述}) \implies \neg \, (\text{Kelvin表述})$。
			
			\blankline
			
			Clausius表述 $\implies$ Kelvin表述:
			
			同样采用反证法,即证明 $\neg \, (\text{Kelvin表述}) \implies \neg \, (\text{Clausius表述})$。
			
			假设Kelvin表述不成立,就可以在不产生其他影响的前提下,从单一热源 $T_\text{H}$ 吸热 $Q_\text{H}$ 并将其完全转化为有用功 $W = Q_\text{H}$。它可以推动Carnot制冷机从低温热源 $T_\text{CH}$ 吸收 $Q_\text{C}$ 的热量并传给高温热源 $Q_\text{H} + Q_\text{C}$ 的热量。净结果是热量 $Q_\text{C}$ 从低温热源传给了高温热源,却没有产生其他影响,这就违背了Clausius表述。因此原假设不成立,即有 $\neg \, (\text{Kelvin表述}) \implies \neg \, (\text{Clausius表述})$。
		\end{myProof}
		
		热力学第二定律的核心内容可以概括为:自然界一切热现象过程都是不可逆的。
		
\section{热力学第二定律的数学解释;熵}
	\subsection{Carnot定理}
		\begin{myThm}{Carnot定理}
			工作于两个确定温度之间的所有热机中,可逆热机效率最高。
		\end{myThm}
		
		设两个热机A、B工作于高温热源 $\th_\text{H}$ 和低温热源 $\th_\text{C}$ 之间\footnote{
			这里用 $\th$ 表示温度,而不是像前文一样使用 $T$,原因见下一小节。
		},它们分别从 $\th_\text{H}$ 吸收 $Q_\text{H,\,A}$ 与 $Q_\text{H,\,B}$ 的热量,向 $\th_\text{C}$ 放出 $Q_\text{C,\,A}$ 与 $Q_\text{C,\,B}$ 的热量,并对外做功 $W_\text{A}$ 与 $W_\text{B} \,$\footnote{
			前文用撇号表示系统(热机)对外界做功,这里方便起见直接用 $W$。但需注意,$W_\text{A}$ 与 $W_\text{B}$ 均大于 $0$。
		}。根据定义,其效率分别为
		\begin{equation}
			\h_\text{A} = \frac{W_\text{A}}{Q_\text{H,\,A}} \comma \, \h_\text{B} = \frac{W_\text{B}}{Q_\text{H,\,}} \fullstop
		\end{equation}
		设A是一个可逆热机,因此我们把 $\h_\text{A}$ 写成 $\h_\text{rev,\,A}$。根据Carnot定理,有
		\begin{equation}
			\h_\text{rev,\,A} \geqslant \h_\text{B} \fullstop
		\end{equation}
		
		\begin{myProof}
			下面利用反证法证明Carnot定理,即假设 $\h_\text{rev,\,A} < \h_\text{B}$。因此
			\begin{equation}
				\frac{W_\text{A}}{Q_\text{H,\,A}} < \frac{W_\text{B}}{Q_\text{H,\,B}} \fullstop
			\end{equation}
			令A、B从高温热源 $\th_\text{H}$ 处吸收相同的热量,即 $Q_\text{H,\,A} = Q_\text{H,\,B}$,那么就有 $W_\text{A} < W_\text{B}$。因为A是可逆热机,所以不妨让B热机输出功的一部分 $W_\text{A}$ 推动A热机逆向运行(此时A就是一个制冷机)。此时,B热机还可以输出功 $W_\text{B} - W_\text{A}$。
			
			根据热力学第一定律,有
			\begin{mySubEq}
				\begin{empheq}[left=\empheqlbrace]{align}
					& W_\text{A} = Q_\text{H,\,A} - Q_\text{C,\,A} \comma \\
					& W_\text{B} = Q_\text{H,\,B} - Q_\text{C,\,B} \comma
				\end{empheq}
			\end{mySubEq}
			因此
			\begin{equation}
				W_\text{B} - W_\text{A} = Q_\text{C,\,A} - Q_\text{C,\,B} \fullstop
			\end{equation}
			
			若A、B联合运行,其净结果便是从低温热源 $\th_\text{C}$ 处吸收 $Q_\text{C,\,A} - Q_\text{C,\,B}$ 的热量,并对外做了 $W_\text{B} - W_\text{A}$ 的功,即在不产生其他影响的情况下完全把热转化为了功。这显然违背了热力学第二定律的Kelvin表述。因此原假设不成立,于是Carnot定理得证。
		\end{myProof}
		
		由Carnot定理,可以得到如下推论:
		\begin{myThm*}
			所有工作于两个确定温度之间的可逆热机效率均相等。
		\end{myThm*}
		
	\subsection{热力学温标}
		\emphA{温标(scale of temperature)},是以量化数值,配以温度单位来表示温度的方法。它包含三个要素:
		\begin{myEnum2}
			\item \emphB{测温质}与\emphB{测温参量};
			\item 测温参量与温度的\emphB{函数关系};
			\item \emphB{温度标准点}的选定。
		\end{myEnum2}
		
		常用的经验温标有摄氏温标、华氏温标等。利用理想气体状态方程,可以定义\emphA{理想气体温标}:
		\begin{equation}
			T \eqdef \frac{1}{n R} \lim\limits_{p \approach 0} p V \comma
		\end{equation}
		同时需要规定水的三相点温度 $T_\text{tr} \eqdef \SI{273.16}{\kelvin}$。
		
		在 \secref{SEC_理想气体与Carnot循环_热力学第二定律} \subsecref{SUBSEC_Carnot循环} 中,我们使用 $T$ 表示温度。实际上,那里的“温度”是用\emphB{理想气体温标}表示的值。
		
		\blankline
		
		根据Carnot定理,可逆热机的效率只与两个热源的温度有关,而与工作物质的性质、吸放热多少、做功多少均无关。因此,可逆热机的效率是两个温度 $\th_\text{H}$、$\th_\text{C}$ 的\emphB{普适函数}。根据定义,热机的效率
		\begin{equation}
			\h = \frac{W}{Q_\text{H}} = 1 - \frac{Q_\text{C}}{Q_\text{H}} \fullstop
		\end{equation}
		因此有
		\begin{equation}
			\frac{Q_\text{C}}{Q_\text{H}} = F(\th_\text{H}, \, \th_\text{C}) \comma
		\end{equation}
		其中的 $F(\th_\text{H}, \, \th_\text{C})$ 是 $\th_\text{H}$ 与 $\th_\text{C}$ 的普适函数。
		
		下面证明
		\begin{equation}
			F(\th_\text{H}, \, \th_\text{C}) = \frac{f(\th_\text{C})}{f(\th_\text{H})} \comma
		\end{equation}
		其中的 $f$ 是另一个普适函数。%TODO:20160322 热力学温标证明
		
		由式~\eqref{EQ_CARNOT_CYCLE_EFFICIENCY_WITH_IDEAL_GAS},理想气体Carnot热机效率为
		\begin{equation}
			1 - \frac{T^*_\text{C}}{T^*_\text{H}} \comma
		\end{equation}
		这里用带 $^*$ 的 $T$ 表示理想气体温标下的温度。这与式 是相同的,即温度尺度相同。又因为理想气体温标也规定在水的三相点处 $T^*_\text{tr} = \SI{273.16}{\kelvin}$,因此,理想气体温标与热力学温标是相同的。
		
	\subsection{Clausius不等式}
		根据Carnot定理,工作于两个确定温度之间的所有热机,其效率均满足
		\begin{equation}
			\h = 1 - \frac{Q_2}{Q_1} \leqslant 1 - \frac{T_2}{T_1} \comma \footnote{
				以后均直接用 $T$ 表示温度。
			}
		\end{equation}
		对于可逆热机,取等号;对于不可逆热机,则取小于号。
		
		上式稍作变形,可得
		\begin{align}
			&\mathrel{\phantom{\implies}} \frac{Q_2}{Q_1} \geqslant \frac{T_2}{T_1} \\
			&\implies \frac{Q_1}{T_1} - \frac{Q_2}{T_2} \leqslant 0 \fullstop
		\end{align}
		约定 $Q$ 始终表示吸收的热量,则放热应写作 $-Q$。于是
		\begin{equation}
			\frac{Q_1}{T_1} + \frac{Q_2}{T_2} \leqslant 0 \fullstop
		\end{equation}
		假设系统先后与温度分别为 $T_1, \, T_2 \, \dots, \, T_n$ 的 $n$ 个热源接触,又分别吸热 $Q_1, \, Q_2 \, \dots, \, Q_n$,则可以证明\emphA{Clausius不等式}:
		\begin{equation}
			\iTonSum \frac{Q_i}{T_i} \leqslant 0 \fullstop
		\end{equation}
		
		\begin{myProof}
			设有程%TODO:20160322 证明过程没写
		\end{myProof}
		
		在 $n \approach \infty$ 的极限下,Clausius不等式过渡到积分形式:
		\begin{equation} \label{EQ_CLAUSISU_INEQUALITY_IN_INTEGRAL}
			\lim\limits_{n \approach \infty} \iTonSum \frac{Q_i}{T_i} \leqslant 0 \approach \oint \frac{\db Q}{T} \leqslant 0 \fullstop
		\end{equation}
		
	\subsection{熵的定义}
		对于可逆循环,根据式~\eqref{EQ_CLAUSISU_INEQUALITY_IN_INTEGRAL},有
		\begin{equation}
			\oint \frac{\db Q_\text{rev}}{T} = 0 \fullstop
		\end{equation}
		如图%TODO:20160323 图片
		可以表示成两段路径之和:
		\begin{equation}
			\underset{C_1\phantom{M}}{\int_{(P_0)}^{(P)}} \, \frac{\db Q_\text{rev}}{T}
			+ \underset{C_2\phantom{M}}{\int_{(P)}^{(P_0)}} \, \frac{\db Q_\text{rev}}{T} = 0 \comma
		\end{equation}
		即
		\begin{equation}
			\underset{C_1\phantom{M}}{\int_{(P_0)}^{(P)}} \, \frac{\db Q_\text{rev}}{T}
			= \underset{C_2\phantom{M}}{\int_{(P_0)}^{(P)}} \, \frac{\db Q_\text{rev}}{T} = \const
		\end{equation}
		可以看出,$\db Q_\text{rev} / T$ 是一个与路径无关的量。由此,定义一个新的状态函数——\emphA{熵(entropy)}:
		\begin{equation}
			S - S_0 = \int_{(P_0)}^{(P)} \, \frac{\db Q_\text{rev}}{T} \fullstop
		\end{equation}
		
	\subsection{不可逆过程的数学表述}
		\begin{myEnum1}
			\myItem{初终态均是平衡态}
				根据Clausius不等式,有
				\begin{align}
					&\mathrel{\phantom{\implies}} \underset{\text{irrev} + \text{rev}}{\oint} \frac{\db Q}{T} < 0 \notag \\
					&\implies \int_{(P_0)}^{(P)} \, \frac{\db Q_\text{irrev}}{T} + \int_{(P)}^{(P_0)} \, \frac{\db Q_\text{rev}}{T} < 0 \notag \\
					&\implies S - S_0 > \int_{(P_0)}^{(P)} \, \frac{\db Q_\text{irrev}}{T} \fullstop
				\end{align}
				
			\myItem{初终态均是非平衡态}
				采用\emphB{局域平衡近似},仍旧可以推得
				\begin{equation}
					S - S_0 > \int_{(P_0)}^{(P)} \, \frac{\db Q_\text{irrev}}{T} \fullstop
				\end{equation} %TODO:20160323 局域平衡近似的证明
		\end{myEnum1}
		
		\blankline
		
		把对可逆过程与不可逆过程的表述合起来,就有
		\begin{equation} \label{EQ_2ND_LAW_IN_INTEGRAL_FORM}
			\incr S = S - S_0 \geqslant \int_{(P_0)}^{(P)} \, \frac{\db Q}{T} \semicomma
		\end{equation}
		写成微分形式,为
		\begin{boxedEq} \label{EQ_2ND_LAW_IN_DIFFERENTIAL_FORM}
			\dd S \geqslant \frac{\db Q}{T}
		\end{boxedEq}
		以上两式中,“$=$”适用于可逆过程,“$>$”适用于不可逆过程。这两式实际上便是热力学第二定律的数学表述。
		
	\subsection{熵的性质}
		这里小结一下熵的性质。
		
		\begin{myEnum2}
			\item 熵是\emphB{状态函数}。
			\item 熵是\emphB{广延量}。
			\item 对微小的\emphB{可逆}过程,$\dd S = \db Q / T$。因此有
			\begin{equation} \label{EQ_dQ=TdS}
				\db Q = T \dd S \fullstop
			\end{equation}
			对于\emphB{绝热}过程,有 $\db Q = 0$,因此
			\begin{equation} \label{EQ_dS=0_FOR_ADIABATIC_REVERSIBLE_PROCESS}
				\dd S = 0 \fullstop
			\end{equation}
		\end{myEnum2}%TODO:20160323 卡诺循环的TS表述
		
	\subsection{热力学基本方程}
		热力学第一定律式~\eqref{EQ_1ST_LAW_IN_DIFFERENTIAL_FORM}:
		\begin{equation}
			\dd U = \db Q + \db W \semicomma
		\end{equation}
		由热力学第二定律,得可逆过程微热量的表达式 \eqref{EQ_dQ=TdS}:
		\begin{equation}
			\db Q = T \dd S \semicomma
		\end{equation}
		微功的一般表示式~\eqref{EQ_GENERAL_WORK}:
		\begin{equation}
			\db W = \iTonSum Y_i \dd y_i \fullstop
		\end{equation}
		联立以上三式,可得
		\begin{equation} \label{EQ_FUNDAMENTAL_EQUATION_OF_THERMODYNAMICS}
			\dd U = T \dd S + \iTonSum Y_i \dd y_i \fullstop
		\end{equation}
		这就是\emphA{热力学基本微分方程}。
		
		对于 $p\text{-}V\text{-}T$ 系统,上式可简化为
		\begin{equation} \label{EQ_FUNDAMENTAL_EQUATION_FOR_PVT_SYSTEM}
			\dd U = T \dd S - p \dd V \fullstop
		\end{equation}
		
		\begin{myExample}[理想气体的熵]
			下面推导不同过程下理想气体的熵。
			\begin{myEnum1}
				\myItem{等容过程}
					根据式~\eqref{EQ_dU=Cv*dT_IDEAL_GAS},有
					\begin{equation}
						\dd U = C_V \dd T \fullstop
					\end{equation}
					根据热力学基本微分方程式~\eqref{EQ_FUNDAMENTAL_EQUATION_FOR_PVT_SYSTEM},
					\begin{align}
						&\mathrel{\phantom{\implies}} T \dd S = \dd U + p \dd V \notag \\
						&\implies \dd S = \frac{\dd U}{T} + \frac{p \dd V}{T} \notag \\
						&\implies S = \frac{C_V}{T} \dd T + \frac{}{}
					\end{align}%TODO:20160323 有问题?
				
				\myItem{等压过程}
					没写%TODO:20160330 没写
				\myItem{等温过程}
			\end{myEnum1}
		\end{myExample}
	
\section{熵增加原理;最大功} \label{SEC_熵增加原理与最大功}
	\subsection{熵增加原理}
		根据热力学第二定律[式~\eqref{EQ_2ND_LAW_IN_INTEGRAL_FORM}],
		\begin{equation}
			\incr S \geqslant \int_{\text{I}}^{\text{II}} \frac{\db Q}{T} \fullstop
		\end{equation}
		对于\emphB{绝热}过程(或孤立体系),有 $\db Q = 0$。因此
		\begin{equation} \label{EQ_PRINCIPLE_OF_ENTROPY_INCREASE}
			\incr S \geqslant 0 \fullstop \footnote{
				注意与 \eqref{EQ_dS=0_FOR_ADIABATIC_REVERSIBLE_PROCESS} 式对比,它还要求\emphA{可逆}过程。
			}
		\end{equation}
		这就是\emphA{熵增加原理},它说明绝热体系的熵永不减少。
	\subsection{不可逆过程的熵变}
		没写%TODO:20160323 没写
	\subsection{最大功}
		根据热力学第一定律[式~\eqref{EQ_1ST_LAW_IN_DIFFERENTIAL_FORM}],
		\begin{equation}
			\dd U = \db Q + \db W \fullstop
		\end{equation}
		令 $\db W' = - \db W$ 为系统对外界做的功,则
		\begin{equation} \label{EQ_dU=dQ-dW'_IN_SECTION_MAX_WORK}
			\db W' = \db Q - \dd U \fullstop
		\end{equation}
		根据热力学第二定律[式~\eqref{EQ_2ND_LAW_IN_DIFFERENTIAL_FORM}],
		\begin{equation}
			\db Q \leqslant T_\text{e} \dd S \fullstop
		\end{equation}
		代入式~\eqref{EQ_dU=dQ-dW'_IN_SECTION_MAX_WORK},可得
		\begin{equation}
			\db W' \leqslant T_\text{e} \dd S - \dd U \fullstop
		\end{equation}
		因此系统对外做的\emphA{最大功}为
		\begin{equation}
			\db W'_{\text{max}} = \db W'_{\text{rev}} = T \dd S - \dd U \comma
		\end{equation}
		这里的 $T = T_\text{e}$ 为系统的温度(因为是可逆过程)。
		
		对于不可逆过程,显然有
		\begin{equation}
			\db W'_{\text{irrev}} < \db W'_{\text{rev}} \fullstop
		\end{equation}
		
		\begin{myExample}[水的混合]
			两杯等量的水初始温度分别为 $T_1$、$T_2$。在等压、绝热条件下将它们混合均匀,求该过程的熵变。%TODO:20160323 没写
			%TODO:20160330 T取平均值:假设热容为常数
		\end{myExample}
		
		\begin{myExample}[制冷机所需的最小功]
			两物体初始温度均为 $T_1$。一台制冷机工作于其间,使一物体温度升高至 $T_2$。假设这是一个等压过程,并且不考虑相变。证明:制冷机所需的最小功
			\begin{equation}
				W_{\text{min}} = C_p \left( \frac{T_1^2}{T_2} + T_2 - 2 T_1 \right) \fullstop%TODO:20160323 没写
			\end{equation}
		\end{myExample}
	
\section{自由能与Gibbs函数}
%	根据熵增加原理[式~\eqref{EQ_PRINCIPLE_OF_ENTROPY_INCREASE}],对于孤立系统,有
%	\begin{equation}
%		\incr S \geqslant 0 \fullstop
%	\end{equation}
	\subsection{自由能}
		考虑这样的\emphB{等温过程}:热源维持恒定温度 $T$;系统初终态温度 $T_1$、$T_2$ 与热源温度相同,即 $T_1 = T_2 =T$。对于可逆过程,在全程中系统温度均为 $T$;而对于不可逆过程,仅满足 $T_1 = T_2 =T$。
		
		由Clausius不等式%[\eqref{}]
		\begin{equation}
			\incr S = S_2 - S_1 \geqslant \int_{\text{I}}^{\text{II}} \frac{\db Q}{T} = \frac{1}{T} \int_{\text{I}}^{\text{II}} \db Q = \frac{Q}{T}\comma 
		\end{equation}
		即
		\begin{equation}
			Q \leqslant T (S_2 - S_1) \fullstop
		\end{equation}
		根据热力学第一定律[式~\eqref{EQ_1ST_LAW_IN_INTEGRAL_FORM}],
		\begin{equation}
			U_2 - U_1 = W + Q \comma
		\end{equation}
		因此
		\begin{align}
			-W &= (U_1 - U_2) + Q \notag \\
			&\leqslant (U_1 - U_2) - T (S_2 - S_1) \notag \\
			&=(U_1 - T S_1) - (U_2 - T S_2) \fullstop
		\end{align}
		定义\emphA{自由能} $F = U - T S$,则
		\begin{equation}
			-W \leqslant F_1 - F_2 \fullstop
		\end{equation}
		
		如果该过程除了保持等温,还保持等容,即 $W = 0$,则有%TODO:20160325 怎么会有等温等容?
		\begin{equation}
			\incr F = F_2 - F_1 \leqslant 0 \comma
		\end{equation}
		这说明在等温等容过程中,系统向自由能减小的方向前进。
		
		自由能具有以下的性质:
		\begin{myEnum2}
			\item 态函数 %TODO:没写
		\end{myEnum2}
	\subsection{Gibbs函数}
		考虑等温等压过程
			
		\chapter{均匀系统的平衡特性}
			\section{Maxwell关系} \label{SEC_Maxwell关系}
	\subsection{热力学函数的全微分}
		\begin{myEnum1}
			\myItem{$U$ 的全微分}
				根据热力学基本微分方程[见式~\eqref{EQ_FUNDAMENTAL_EQUATION_FOR_PVT_SYSTEM}]
				\begin{equation} \label{EQ_dU=TdS-pdV_IN_MAXWELL_RELATION}
					\dd U = T \dd S + p \dd V \comma
				\end{equation}
				可知内能 $U$ 是 $S$ 和 $V$ 的函数,即
				\begin{equation}
					U = U(S, \, V) \fullstop
				\end{equation}
				因此
				\begin{equation}
					\dd U = \myPartial{U}{S}{V} \dd S + \myPartial{U}{V}{S} \dd V \fullstop
				\end{equation}
				与式~\eqref{EQ_dU=TdS-pdV_IN_MAXWELL_RELATION} 进行比较,可得
				\begin{braceEq}
					& T = \myPartial{U}{S}{V} \comma  \label{EQ_T_IN_PARTIAL_DIFFERENTIAL_OF_U} \\
					& p = -\myPartial{U}{V}{S}  \label{EQ_p_IN_PARTIAL_DIFFERENTIAL_OF_U} \fullstop
				\end{braceEq}
				$U$ 的二阶偏导数与其先后次序无关,即
				\begin{equation}
					\frac{\pd^{\:2} \! U}{\pd V \pd S} = \frac{\pd^{\:2} \! U}{\pd S \pd V}%HACK:20160330 偏微分空格
				\end{equation}
				或
				\begin{equation}
					\left[ \frac{\pd}{\pd V} \myPartial{U}{S}{V}\right]_S
					= \left[ \frac{\pd}{\pd S} \myPartial{U}{V}{S}\right]_V \fullstop
				\end{equation}
				利用该式,对式~\eqref{EQ_T_IN_PARTIAL_DIFFERENTIAL_OF_U} 和式~\eqref{EQ_p_IN_PARTIAL_DIFFERENTIAL_OF_U} 两边分别求导,可得
				\begin{equation} \label{EQ_MAXWELL_RELATION_FROM_U}
					\myPartial{T}{V}{S} = -\myPartial{p}{S}{V} \fullstop
				\end{equation}
				
			\myItem{$H$ 的全微分}
				因为
				\begin{equation}
					H \eqdef U + p V \comma
				\end{equation}
				所以
				\begin{align}
					\dd H &= T \dd S - p \dd V + \dd \; (p V) \notag \\
					&= T \dd S - p \dd V + (V \dd p + p \dd V) \notag \\
					&= T \dd S + V \dd p \fullstop
				\end{align}%HACK:20160330 微分d后接空格,间距有问题
				这说明焓 $H$ 是 $S$ 和 $p$ 的函数,即
				\begin{equation}
					H = H(S, \, p) \fullstop
				\end{equation}
				因此
				\begin{equation}
					\dd H = \myPartial{H}{S}{p} \dd S + \myPartial{H}{p}{S} \dd p \fullstop
				\end{equation}
				进而
				\begin{braceEq}
					& T = \myPartial{H}{S}{p} \comma \\
					& V = \myPartial{H}{p}{S} \fullstop
				\end{braceEq}
				与上文类似,对它们两边分别求导,并利用混合偏导数定理,就得到
				\begin{equation} \label{EQ_MAXWELL_RELATION_FROM_H}
					\myPartial{T}{p}{S} = \myPartial{V}{S}{p} \fullstop
				\end{equation}
				
				\myItem{$F$ 的全微分}
				因为
				\begin{equation}
					F \eqdef U - T S \comma
				\end{equation}
				所以
				\begin{equation} \label{EQ_dF=-SdT-pdV_IN_MAXWELL_RELATION}
					\dd F = -S \dd T - p \dd V \fullstop
				\end{equation}
				这说明自由能 $F$ 是 $T$ 和 $V$ 的函数,即
				\begin{equation}
					F = F(T, \, V) \fullstop
				\end{equation}
				因此
				\begin{equation}
					\dd F = \myPartial{F}{T}{V} \dd T + \myPartial{F}{V}{T} \dd V \fullstop
				\end{equation}
				进而
				\begin{braceEq}
					& S = -\myPartial{F}{T}{V} \comma \\
					& p = -\myPartial{F}{V}{T} \fullstop
				\end{braceEq}
				两边分别求导,得
				\begin{equation} \label{EQ_MAXWELL_RELATION_FROM_F}
					\myPartial{S}{V}{T} = \myPartial{p}{T}{V} \fullstop
				\end{equation}
			
			\myItem{$G$ 的全微分}
				因为
				\begin{equation}
					G \eqdef U - T S + p V \comma
				\end{equation}
				所以
				\begin{equation}
				\dd G = -S \dd T + V \dd p \fullstop
				\end{equation}
				这说明Gibbs自由能是 $T$ 和 $p$ 的函数,即
				\begin{equation}
					G = G(T, \, p) \fullstop
				\end{equation}
				因此
				\begin{equation}
					\dd H = \myPartial{G}{G}{p} \dd T + \myPartial{G}{p}{T} \dd T \fullstop
				\end{equation}
				进而
				\begin{braceEq}
					& S = -\myPartial{G}{T}{p} \comma \\
					& V = \myPartial{G}{p}{T} \fullstop
				\end{braceEq}
				两边分别求导,得
				\begin{equation} \label{EQ_MAXWELL_RELATION_FROM_G}
				\myPartial{S}{p}{T} = -\myPartial{V}{T}{p} \fullstop
				\end{equation}
		\end{myEnum1}
		
		\blankline
		式~\eqref{EQ_MAXWELL_RELATION_FROM_U}、\eqref{EQ_MAXWELL_RELATION_FROM_H}、\eqref{EQ_MAXWELL_RELATION_FROM_F} 和 \eqref{EQ_MAXWELL_RELATION_FROM_G} 称为\emphA{Maxwell关系}。
		
	\subsection{Legendre变换}
		\emphA{Legendre变换}是指等式
		\begin{equation}
			x \dd y = \dd \; (x y) - y \dd x \comma
		\end{equation}
		它把变量从 $x$ 变为了 $y$。
		
		在经典力学中,Legendre变换被用来从Lagrange表述导出Hamilton表述:
		\begin{equation}
			\scL (q_\a, \, \dot{q_\a}) \quad \rightarrow \quad \scH (q_\a, \, p_\a) = \sum_\a \dot{q_\a} p_\a - \scL (q_\a, \, \dot{q_\a}) \fullstop
		\end{equation}
		
		对式~\eqref{EQ_dU=TdS-pdV_IN_MAXWELL_RELATION} 应用Legendre变换
		\begin{equation}
			T \dd S = \dd \; (T S) - S \dd T
		\end{equation}
		并移项,就得到
		\begin{equation}
			\dd U - \dd \; (T S) = \dd \; (U - T S)= \dd F = -S \dd T + p \dd V \fullstop
		\end{equation}
		这也就是用\emphB{自由能}表示的热力学基本方程,即式~\eqref{EQ_dF=-SdT-pdV_IN_MAXWELL_RELATION}。由此,利用偏微分关系稍做计算,就得到了Maxwell关系 \eqref{EQ_MAXWELL_RELATION_FROM_F} 式。用同样的手法,也可以得到其他几个Maxwell关系。
		
		\begin{myTable}[H]{MM>{$}l<{$}}{Maxwell关系}{TAB_MAXWELL_RELATION}
			\toprule
			\text{\emphA{基本微分方程}} & \text{\emphA{自然变量}} & \text{\emphA{Maxwell关系}} \\
			\midrule
			\dd U = T \dd S - p \dd V & (S, \, V) & \myPartialDisplay{T}{V}{S} = -\myPartialDisplay{p}{S}{V} \\
			\dd H = \dd \; (U + p V) = T \dd S + V \dd p & (S, \, p) & \myPartialDisplay{T}{p}{S} = \myPartialDisplay{V}{S}{p} \\
			\dd F = \dd \; (U - T S) = -S \dd T - p \dd V & (T, \, V) & \myPartialDisplay{S}{V}{T} = \myPartialDisplay{p}{T}{V} \\
			\dd G = \dd \; (U - T S + p V) = -S \dd T + V \dd p & (T, \, p) & \myPartialDisplay{S}{p}{T} = -\myPartialDisplay{V}{T}{p} \\
			\bottomrule
		\end{myTable}%FIXME:20160330 表格格式
		
		表~\ref{TAB_MAXWELL_RELATION} 总结了上文的推导结果。其中的“\emphA{自然变量}”指经Legendre变换后的自由变量,它们直接出现在热力学基本方程中。
		
		热力学变量(或函数)[见 \secref{SEC_平衡态及其描述}\subsecref{SUBSEC_平衡态的描述}]可分为两种:\emphA{可测量量}和\emphA{不可测量量}。物态方程中的 $p$、$V$、$T$,与物态方程有关的 $\a$、$\b$、$\k_T$,以及热容,都是可测量量;而 $U$、$S$、$H$、$F$、$G$ 等状态函数,以及 $(\pd S / \pd V)_T$、$(\pd H / \pd p)_T$ 等偏微分,都是不可测量量。Maxwell关系的作用,就是把不可测量量用可测量量来表示。
		
	\subsection{简单应用} \label{SUBSEC_简单应用_OF_MAXWELL关系}
		\begin{myExample} \label{EG_C_p-C_V}
			计算 $C_p - C_V$。\footnote{
				注意与式~\eqref{EQ_C_p-C_V_FOR_IDEAL_GAS} 对比,那里要求理想气体。
			}%TODO:20160330 式1.4.8有问题
			
			根据 \secref{SEC_热容与焓;理想气体的性质}\subsecref{SUBSEC_热容与焓} 中的推导,
			\begin{braceEq}
				C_p &= \myPartial{H}{T}{p} \comma \\
				C_V &= \myPartial{U}{T}{V} \fullstop
			\end{braceEq}
			利用 $\db Q = T \dd S$,可得%TODO:20160330 为什么dQ=TdS(可逆?)
			\begin{braceEq}
				C_p &= T \myPartial{S}{T}{p} \comma \label{EQ_C_p_IN_T_AND_S} \\
				C_V &= T \myPartial{S}{T}{V} \fullstop \label{EQ_C_V_IN_T_AND_S} 
			\end{braceEq}%TODO:20160330 推导过程
			把 $S = S(T, \, p)$ 看成复合函数的形式,即 $S[T, \, V(T, \, p)]$,因此
			\begin{equation}
				\myPartial{S}{T}{p} = \myPartial{S}{T}{V} + \myPartial{S}{V}{T} \myPartial{V}{T}{p} \fullstop
			\end{equation}
			于是
			\begin{align}
				C_p - C_V &= T \left[ \myPartial{S}{T}{p} - \myPartial{S}{T}{V} \right] \notag \\
				&= T \myPartial{S}{V}{T} \myPartial{V}{T}{p} \notag \\%FIXME:20160401 强调
				&= T \myPartial{p}{T}{V} \myPartial{V}{T}{p} \fullstop
				\myTagNumbering{Maxwell关系}  \label{EQ_C_p-C_V_WITH_MAXWELL_RELATION_PART_1}
			\end{align}
			根据\emphB{偏导数三乘积法则} \eqref{EQ_CYCLIC_RELATION} 式,有
			\begin{equation}
				\myPartial{p}{T}{V} = -\myPartial{V}{T}{p} \myPartial{p}{V}{T} \semicomma
			\end{equation}
			根据\emphB{倒数关系} \eqref{EQ_RECIPROCITY_RELATION} 式,有
			\begin{equation}
				\myPartial{V}{T}{p} = \myPartial{T}{V}{p}^{-1} \fullstop
			\end{equation}
			把以上两式代入式~\eqref{EQ_C_p-C_V_WITH_MAXWELL_RELATION_PART_1},得
			\begin{align}
				C_p - C_V &= T \myPartial{p}{T}{V} \myPartial{V}{T}{p} \notag \\
				&= T \left[ -\myPartial{V}{T}{p} \myPartial{p}{V}{T} \right] \myPartial{V}{T}{p} \notag \\
				&= -T \myPartial{V}{T}{p}^2 \myPartial{V}{p}{T}^{-1} \fullstop \label{EQ_C_p-C_V_WITH_MAXWELL_RELATION_PART_2}
			\end{align}
			再把\emphB{膨胀系数}、\emphB{等温压缩系数}的定义
			\begin{braceEq}
				&\a = \frac{1}{V} \myPartial{V}{T}{p} \comma \\
				&\k_T = -\frac{1}{V} \myPartial{V}{p}{T}
			\end{braceEq}
			代入式~\eqref{EQ_C_p-C_V_WITH_MAXWELL_RELATION_PART_2},便得到
			\begin{align}
				C_p - C_V &= -T (V \a)^2 \left( -\frac{1}{V \k_T} \right) \notag \\
				&= T V \frac{\a^2}{\k_T} \label{EQ_C_p-C_V_WITH_MAXWELL_RELATION_PART_3} \fullstop
			\end{align}
			这几个量都是可测量量。热力学稳定相中,有 $\k_T > 0$。因此 $C_p$ 始终大于 $C_V$。%FIXME:20160401 后续交叉引用
			
			对于理想气体,有
			\begin{braceEq}
				&\a = \frac{1}{V} \myPartial{V}{T}{p}
				= \frac{1}{V} \left( \frac{\pd}{\pd T} \frac{n R T}{p} \right)_p
				= \frac{n R}{p V} \comma \\
				&\k_T = -\frac{1}{V} \myPartial{V}{p}{T}
				= -\frac{1}{V} \left( \frac{\pd}{\pd p} \frac{n R T}{p} \right)_T
				= \frac{n R T}{p^2 V} \fullstop
			\end{braceEq}
			根据式~\eqref{EQ_C_p-C_V_WITH_MAXWELL_RELATION_PART_3},可知
			\begin{equation}
				C_p - C_V = T V \left( \frac{n R}{p V} \right)^2 \left( \frac{p^2 V}{n R T} \right) = n R \comma
			\end{equation}
			这与式~\eqref{EQ_C_p-C_V_FOR_IDEAL_GAS} 是一致的。
		\end{myExample}
		
		\begin{myExample} \label{EG_pd_U/pd_V_WITH_FIXED_T}
			计算 $(\pd U/ \pd V)_T$。
			
			选取 $T$ 和 $V$ 作为独立变量,则
			\begin{align}
				\dd U &= T \dd S - p \dd V \notag \\
				&= T \left[ \myPartial{S}{T}{V} \dd T + \myPartial{S}{V}{T} \dd T \right] - p \dd V \notag \\
				&= T \myPartial{S}{T}{V} \dd T + \left[ T \myPartial{S}{V}{T} - p \right] \dd V \comma
			\end{align}
			因此
			\begin{align}
				\myPartial{U}{V}{T} &= T \myPartial{S}{V}{T} - p \notag \\
				&= T \myPartial{p}{T}{V} - p \fullstop \label{EQ_pd_U/pd_V_WITH_FIXED_T}
			\end{align}
			这里出现的都是状态变量或是与状态方程直接相关的量,因而也是可测量量。
			
			对于理想气体,$p V = n R T$,因此
			\begin{equation}
				\myPartial{p}{T}{V} = \left( \frac{\pd}{\pd T} \frac{n R T}{V} \right)_V = \frac{n R}{V} = \frac{p}{T} \comma
			\end{equation}
			代入式~\eqref{EQ_pd_U/pd_V_WITH_FIXED_T},可以发现
			\begin{equation}
				\myPartial{U}{V}{T} = 0 \fullstop
			\end{equation}
			这说明 $U = U(T)$,即内能 $U$ 仅是温度 $T$ 的函数,而与体积 $V$ 无关。这就是\secref{SEC_热容与焓;理想气体的性质}\subsecref{SUBSEC_理想气体的性质}中提到的利用状态方程来证明内能只与温度有关的方法。
			
			对于van der Waals气体,其状态方程 [式~\eqref{EQ_VAN_DER_WAALS_GAS_STATE_EQUATION}]为
			\begin{equation}
				\left( p + \frac{n^2 a}{V^2} \right) (V - n b) = n R T \fullstop
			\end{equation}
			在固定 $V$ 的条件下关于 $T$ 求偏导,得
			\begin{equation}
				\myPartial{p}{T}{V} (V - n b) + \left( p + \frac{n^2 a}{V^2} \right) \cdot 0 = n R \comma
			\end{equation}
			即
			\begin{equation}
				\myPartial{p}{T}{V} = \frac{n R}{V - n b} \fullstop
			\end{equation}
			因此
			\begin{equation}
				\myPartial{U}{V}{T} = \frac{n R T}{V - n b} - p = \frac{n^2 a}{V^2} \fullstop
			\end{equation}
			可见van der Waals气体的内能与体积\emphB{有关}。%TODO:20160401 与温度T有关?
		\end{myExample}
		
		\begin{myExample}
			证明绝热压缩系数与等温压缩系数之比
			\begin{equation}
				\frac{\k_\text{a}}{\k_T} = \frac{C_V}{C_p} \fullstop
			\end{equation}%HACK:20160401 绝热过程下标用\text{a}
			
			可逆的绝热过程是等熵过程,因此把 $\k_\text{a}$ 的下标改用“$S$”。%TODO:20160401 为什么可逆
			
			根据定义
			\begin{braceEq}
				\k_S &\eqdef -\frac{1}{V} \myPartial{V}{p}{S} \comma \\
				\k_T &\eqdef -\frac{1}{V} \myPartial{V}{p}{T} \fullstop
			\end{braceEq}
			因此
			\begin{align}
				\frac{\k_S}{\k_T} &= \left. \myPartialDisplay{V}{p}{S} \, \middle/ \, \myPartialDisplay{V}{p}{T} \right. \notag \\
				&= \left. \frac{ \pd \; (V, \, S) }{ \pd \; (p, \, S) } \, \middle/ \, \frac{ \pd \; (V, \, T) }{ \pd \; (p, \, T) } \right. \notag \\
				&= \left. \frac{ \pd \; (V, \, S) }{ \pd \; (V, \, T) } \, \middle/ \, \frac{ \pd \; (p, \, S) }{ \pd \; (p, \, T) } \right. \myTag{见式} \\%FIXME:20160402 交叉引用没写
				&= \left. \myPartialDisplay{S}{T}{V} \, \middle/ \, \myPartialDisplay{S}{T}{p} \right. \notag \\
				&= \frac{C_V / T}{C_p / T} \myTag{见\egref{EG_C_p-C_V}} \\
				&= \frac{C_V}{C_p} \fullstop
			\end{align}
			
			\begin{myProof}
				这里证明几个上面用到的结论:%TODO:偏微分关系的证明
				\begin{myEnum2}
					\item
						\begin{equation} \label{EQ_PARTIAL_DIFFERENTIAL_TO_JACOBI_DET}
							\myPartial{f}{g}{h} = \frac{ \pd \; (f, \, h) }{ \pd \; (g, \, h) } \fullstop
						\end{equation}
					\item
						\begin{equation} \label{EQ_JACOBI_DET_FRACTION}
							\myPartial{f}{g}{h} = \left. \frac{ \pd \; (f, \, h) }{ \pd \; (x, \, y) } \, \middle/ \, \frac{ \pd \; (g, \, h) }{ \pd \; (x, \, y) } \right. \fullstop
						\end{equation}
					\item
						\begin{equation} \label{EQ_INVERSE_JACOBI_DET}
							\frac{ \pd \; (f, \, g) }{ \pd \; (x, \, y) } = \left[ \frac{ \pd \; (x, \, y) }{ \pd \; (f, \, g) } \right]^{-1} \fullstop
						\end{equation}
				\end{myEnum2}
			\end{myProof}
		\end{myExample}
		
\section{Joule-Thomson效应;绝热膨胀与制冷}%FIXME:连字符kerning
	\subsection{节流过程}
		为了研究气体的内能,Joule先采用了气体自由膨胀的方法[见\secref{SEC_热容与焓;理想气体的性质}\subsecref{SUBSEC_理想气体的性质}],但其精度不佳。之后(1852年),他又与Thomson采取了另外的手段,即\emphA{节流过程}。
		
		如图,%TODO:20160402 图片,节流过程
		在一根绝热管中间放置一个多孔塞,使气体无法很快地通过。保持多孔塞两边的压强差,则气体将从其一边不断地流到另一边,该过程就是\emphB{节流过程}。
		
		开始时,气体完全在多孔塞的一边。设其压强、体积、内能分别为 $p_1$、$V_1$ 和 $U_1$。让外界做功,直到气体完全流到多孔塞的另一边。此时,气体的压强、体积、内能分别为 $p_2$、$V_2$ 和 $U_2$。显然,外界做功
		\begin{equation}
			\db W = p_1 V_1 - p_2 V_2 \semicomma
		\end{equation}
		因为是绝热过程,因此 $\db Q = 0$。根据热力学第一定律,
		\begin{equation}
			U_2 - U_1 = \db W + \db Q = p_1 V_1 - p_2 V_2 \comma
		\end{equation}
		即
		\begin{equation}
			U_1 + p_1 V_1 = U_2 + p_2 V_2 \fullstop
		\end{equation}
		这说明节流过程是一个\emphB{等焓过程},即 $H_1 = H_2$。\footnote{
			要注意该过程是一个\emphB{不可逆过程}。
		}
		
		定义\emphA{Joule-Thomson系数}
		\begin{equation}
			\m \eqdef \myPartial{T}{p}{H} \fullstop
		\end{equation}
		根据\emphB{偏导数三乘积法则} \eqref{EQ_CYCLIC_RELATION} 式,可知
		\begin{equation}
			\m = -\myPartial{H}{p}{T} \myPartial{T}{H}{p} \fullstop
			%TODO:20160402 没写,制冷曲线等
		\end{equation}
		
	\subsection{绝热膨胀过程}
		对于绝热过程,选取 $T$ 和 $p$ 作为独立变量,则
		\begin{equation}
			\dd S = \myPartial{S}{T}{p} \dd T+ \myPartial{S}{p}{T} \dd p = 0 \fullstop
		\end{equation}
		因此
		\begin{align}
			\myPartial{T}{p}{S} &= -\myPartial{S}{p}{T} \myPartial{T}{S}{p} \myTag{偏导数三乘积法则} \\
			&= -\left. \myPartialDisplay{S}{p}{T} \, \middle/ \, \myPartialDisplay{S}{T}{p} \right. \myTag{倒数关系} \\
			&= -\frac{T}{C_p} \myPartial{S}{p}{T} \myTag{见式~\eqref{EQ_C_p_IN_T_AND_S}} \\
			&= \frac{T}{C_p} \myPartial{V}{T}{p} \myTag{Maxwell关系} \\
			&= \frac{T V \a}{C_p} > 0  \myTagNumbering{见式~\eqref{EQ_DEFINITION_OF_EXPANSION_COEFFICIENT}} \fullstop
		\end{align}
		随着压强的增大,分子间距离减小,相互作用力增大,因而总能量减少,分子平均动能增加,从而使得温度上升。%TODO:20160402 解释对不对——能量
	\subsection{低温的实现}
	
\section{热力学函数的确定}
\section{特性函数}
\section{热辐射的热力学理论}
	这里所讲的热辐射即\emphB{黑体辐射}。温度升高,电偶极子发生振荡,因而辐射出电磁场(电磁波)。这就是热辐射的来源。
	
	本节内容说明热力学不仅适用于实物粒子,还适用于\emphB{场}。
	
	\subsection{内能密度}
		定义\emphA{内能密度}
		\begin{equation}
			u \eqdef \frac{U}{V}
		\end{equation}
		代表热辐射单位体积的内能。下面证明它是温度 $T$ 的\emphA{普适函数}。
		
		设有A、B两个空腔,其中充满了热辐射。A、B具有相同的温度 $T$。它们之间由一个细管连接,细管中有一个滤波片,它只允许频率在 $\n$ 到 $\n + \dd \n$ 之间的辐射通过。
		
		%TODO:20160406 内能密度证明,图片
		
		因此
		\begin{equation}
			u_\text{A}(\n) = u_\text{B}(\n)
		\end{equation}
		于是
		\begin{equation}
			u_\text{A} = u_\text{B} \fullstop
		\end{equation}
		该结果与A、B两空腔的形状、大小、腔壁材质均无关,这就说明了 $u$ 是 $T$ 的普适函数。因此,热辐射的内能
		\begin{equation} \label{EQ_RADIATION_INTERNAL_ENERGY}
			U = U(T, \, V) = V u(T) \fullstop
		\end{equation}
		
	\subsection{压强公式}
		根据实验结果,辐射压强
		\begin{equation} \label{EQ_RADIATION_PRESSURE}
			p = \frac{1}{3} u \fullstop
		\end{equation}
		该结果也可以通过电磁理论证明得到。
		%TODO:20160406 辐射压强的证明
		
	\subsection{热力学函数}
		与一般的 $p\text{-}V\text{-}T$ 系统一样,对于热辐射,有
		\begin{equation}
			\dd U = T \dd S - p \dd V \fullstop
		\end{equation}
		因此
		\begin{align}
			\myPartial{U}{V}{T} &= T \myPartial{S}{V}{T} - p \notag \\
			&= T \myPartial{p}{T}{V} - p \myTagNumbering{Maxwell关系} \fullstop
		\end{align}
		代入式~\eqref{EQ_RADIATION_INTERNAL_ENERGY} 和式~\eqref{EQ_RADIATION_PRESSURE},可得
		\begin{equation}
			u = T \left( \frac{1}{3} \frac{\dd u}{\dd T} \right) - \frac{u}{3} \comma
		\end{equation}
		即
		\begin{equation}
			\frac{\dd u}{u} = \frac{4 \dd T}{T} \fullstop
		\end{equation}
		积分可得
		\begin{equation}
			u = a T^4 \comma
		\end{equation}
		其中的 $a$ 是积分常数。于是热辐射的内能
		\begin{equation}
			U = a T^4 V \fullstop
		\end{equation}
		显然,$V =0 $ 时也成立 $U = 0$,因此上式不含附加的常数。同时,还可以得到压强与温度的关系:
		\begin{equation} \label{EQ_p_T_RELATION_OF_RADIATION}
			p = \frac{1}{3} u = \frac{1}{3} a T^4 \fullstop
		\end{equation}
		
		对于熵,
		\begin{align}
			\dd S &= \frac{1}{T} (\dd U + p \dd V) \notag \\
			&= \frac{1}{T} \left[ \dd \; (u V) + \frac{1}{3} u \dd V \right] \notag \\
			&= \frac{1}{T} \dd \; (a T^4 V^2) + \frac{1}{3 T} a T^4 V \dd V \notag \\
%			&= \frac{a}{T} (4 T^3 V^2 \dd T + 2 T^4 V \dd V) + \frac{1}{3} a T^3 V \ddV \notag \\
			&= \dd \left( \frac{4}{3} a T^3 V \right) \fullstop
		\end{align}
		积分得
		\begin{equation}
			S = \frac{4}{3} a T^3 V + S_0 \comma
		\end{equation}
		这里的 $S_0$ 是积分常数。当 $V \approach 0$ 时,显然应该有 $S \approach 0$。因此$S_0 = 0$,即
		\begin{equation} \label{EQ_RADIATION_ENTROPY}
			S = \frac{4}{3} a T^3 V \fullstop
		\end{equation}
		
		利用内能和熵的表达式,可以求出其他几个热力学函数:
		\begin{braceEq}
			H &= U + p V = \frac{4}{3} a T^4 V \semicomma \\
			F &= U - T S = -\frac{1}{3} a T^4 V \semicomma \\
			G &= U - T S + p V = 0 \fullstop
		\end{braceEq}
		这里的 $G = 0$ 有重要意义:热辐射的化学势为零,微观上说明光子数不守恒。
		
		根据式~\eqref{EQ_RADIATION_ENTROPY},可知
		\begin{braceEq}
			C_V &= T \myPartial{S}{T}{V} = T \cdot 4 a T^2 V = 3 S \semicomma \\
			C_p &= T \myPartial{S}{T}{p} = \infty \fullstop %TODO:20160413 为什么C_p=无穷
		\end{braceEq}
		$C_p = \infty$,说明若 $p$ 不变,则 $T$ 也将保持不变。
		
		考虑可逆绝热过程,即 $S = \const$。如果选取 $T$ 和 $V$ 作为变量,根据式~\eqref{EQ_RADIATION_ENTROPY},则有
		\begin{equation}
			T^3 V = \const \semicomma
		\end{equation}
		若选取 $p$ 和 $V$ 作为变量,利用式~\eqref{EQ_p_T_RELATION_OF_RADIATION},便可得到
		\begin{equation} \label{EQ_STATE_EQUATION_OF_ADIABATIC_PROCESS_OF_RADIATION}
			p V^{4/3} = \const
		\end{equation}
		该式与理想气体可逆绝热过程方程 \eqref{EQ_STATE_EQUATION_OF_ADIABATIC_PROCESS_IN_p_V} 式($p V^\g = \const$)形式上很类似,故它也被称作“光子气体”。但对于理想气体,$\g = C_p / C_V$,而式~\eqref{EQ_STATE_EQUATION_OF_ADIABATIC_PROCESS_OF_RADIATION}中的 $4/3$ 只是幂指数,与 $\g$ 无关。对于热辐射,$\g$ 实际上等于 $\infty$。
		
		\raggedbottom%FIXME:20160315 交叉引用、脚注每页重新计数失效,必须加上该行
		\pagebreak
			
		\chapter{相变的热力学理论}
			我们所研究的系统是逐渐复杂的。首先是\emphB{独立子体系},如理想气体、Fermi气体、Bose气体等;之后是\emphB{近独立子体系},如准粒子气体;%TODO:20160413 翻译
本章主要探讨\emphB{相互作用体系},它包含多种物态,也会有相变的发生。

\section{热动平衡判据;开系热力学}
	在 \secref{SEC_熵增加原理与最大功}中我们已经证明,对于孤立体系(即绝热过程),熵将不断增加,直至达到极大(平衡)。因此,熵达到极大便可以作为体系达到平衡的判据。
	
	\subsection{平衡判据}
		\begin{myEnum1}
			\myItem{熵判据}
				由上,在 $\vd U = 0, \, \vd V = 0, \, \vd N = 0$ 的前提下,$\vd S = 0, \, \vd^{\:2} \! S <0$ 即说明(孤立)体系达到了平衡。这里的“$\vd\,$”表示\emphB{虚变化},与“$\dd\,$”代表的\emphB{真实变化}有所不同,它与虚功原理中的虚位移是类似的。%HACK:20160420 δS的缩进
				
				平衡分为三种:稳定平衡、亚稳平衡和不稳定平衡。如前所述,熵的极大值对应平衡态。稳定平衡对应其中\emphB{最大}的极大值,而亚稳平衡对应其他较小的极大值,即对于无限小的变动是稳定的,而对于有限的变动则是不稳定的。不稳定平衡对应极小值,虽然有 $\vd S = 0$,但却不满足 $\vd^{\:2} \! S <0$。平衡的稳定性可以用力学类比来理解。重力势能 $E_\text{p}$ 的极小值对应平衡。其最小值对应稳定平衡;但相对极小对于大的扰动是不稳定的,所以对应亚稳平衡。而极大值则对应不稳定平衡,稍有扰动就会偏离。%TODO:20160420 图片
				
				对于 $\vd S = 0$、$\vd^{\:2} \! S =0$ 的临界态,将 $S$ 围绕极值点Taylor作展开,可得
				\begin{equation}
					\tl{\incr\,} S = \vd S + \frac{1}{2!} \vd^{\:2} \! S + \frac{1}{3!} \vd^{\:3} \! S + \frac{1}{4!} \vd^{\:4} \! S + \cdots < 0 \fullstop
				\end{equation}
				式中的“$\tl{\incr\,}$”同样表示虚变动。由于要满足 $S \rightarrow -S$ 的对称性,因此 $\vd^{\:3} \! S = 0$。从而%TODO:20160420 三阶项为何等于零
				\begin{equation}
					\vd^{\:4} \! S < 0 \comma
				\end{equation}
				这就是临界态的平衡判据。
				
			\myItem{自由能判据}
				考虑一个系统与热库(即环境)组成的复合体系,它是一个孤立系。显然,总内能
				\begin{equation}
					U_0 = U_\text{sys} + U_\text{res} = \const \comma
				\end{equation}
				总体积
				\begin{equation}
					V_0 = V_\text{sys} + V_\text{res} = \const
				\end{equation}
				设想体系发生了一个虚变动,则
				\begin{braceEq}[gather]
					\vd U_\text{sys} + \vd U_\text{res} = 0 \comma \\
					\vd V_\text{sys} + \vd V_\text{res} = 0 \fullstop \label{EQ_DELTA_V_SYS+DELTA_V_RES=0}
				\end{braceEq}
				根据熵判据,
				\begin{braceEq}
					\vd S_0 &= \vd \; (S_\text{sys} + S_\text{res}) = 0 \comma \\
					\vd^{\:2} \! S_0 &= \vd^{\:2} (S_\text{sys} + S_\text{res}) < 0 \fullstop
				\end{braceEq}
				根据热力学基本微分方程 \eqref{EQ_FUNDAMENTAL_EQUATION_FOR_PVT_SYSTEM} 式,可得
				\begin{equation}
					\vd U_\text{res} = T_\text{res} \vd S_\text{res} + p_\text{res}  \vd V_\text{res} \fullstop \footnote{
						已经把式~\eqref{EQ_FUNDAMENTAL_EQUATION_FOR_PVT_SYSTEM} 中的微分“$\dd\,$”改成了变分“$\vd\,$”的形式。
					}
				\end{equation}
				
				在系统温度、体积均不变(即 $\vd T_\text{sys} = 0$、$\vd V_\text{sys} = 0$)的情形下,根据式~\eqref{EQ_DELTA_V_SYS+DELTA_V_RES=0} 可知 $\vd V_\text{res} = 0$。由于是平衡态,又有 $T_\text{res} = T_\text{sys}$。因此
				
				\begin{align}
					\vd F_\text{sys} &= \vd \; (U_\text{sys} - T_\text{sys} S_\text{sys}) \\
					&= \vd U_\text{sys} - T_\text{sys} \vd S_\text{sys} \\
					&= -\vd U_\text{res} + T_\text{res} \vd S_\text{res} \\
					&= -p_\text{res}  \vd V_\text{res} \\
					&= -p_\text{res}  \vd V_\text{sys} = 0 \comma \\
					\vd^{\:2} \! F_\text{sys} &= %TODO:20160420 推导过程
				\end{align}
				
			\myItem{Gibbs函数判据}
			\myItem{内能判据}
		\end{myEnum1}
		
	\subsection{开系热力学}
		所谓“开系”,即\emphA{开放系(open system)},它指粒子数可变且有能量交换的系统。
		
		对于封闭系,我们已经知道
		\begin{equation}
			\dd G = -S \dd T + V \dd p \fullstop
		\end{equation}
		而对于开放系,需引进\emphA{化学势} $\m$,其定义为
		\begin{equation}
			\m \eqdef \myPartial{G}{n}{T, \, p} \fullstop \footnote{这里的$n$指物质的量,而非粒子数。定义$\m \eqdef (\pd G / \pd N)_{T,\,p}$ 当然也可以,不过我们只采用第一种定义。}
		\end{equation}
		因此,
		\begin{equation}
			\dd G = -S \dd T + V \dd p + \m \dd n \fullstop
		\end{equation}
		利用Legendre变换,可得粒子数可变系统的热力学基本方程:
		\begin{braceEq}
			\dd U &= T \dd S - p \dd V + \m \dd n \comma \\
			\dd H &= T \dd S + V \dd p + \m \dd n \comma \\
			\dd F &= -S \dd T - p \dd V + \m \dd n \fullstop
		\end{braceEq}
		于是可以得出 $\m$ 的几个等价定义:
		\begin{equation}
			\m = \myPartial{U}{n}{S, \, V} = \myPartial{H}{n}{S, \, p} = \myPartial{F}{n}{T, \, V} = \myPartial{G}{n}{T, \, p} \fullstop
		\end{equation}
		
		Gibbs函数 $G(T,\,p,\,n)$ 是广延量。因此可定义\emphA{摩尔Gibbs函数} $G_\text{m}$,使得
		\begin{equation}
			G(T,\,p,\,n) = n G_\text{m} (T,\,p) \fullstop
		\end{equation}
		因此,
		\begin{equation}
			\m \eqdef \myPartial{G}{n}{T, \, p} = \left[ \frac{\pd}{\pd n} (n G_\text{m}) \right]_{T, \, p} = G_\text{m} \fullstop
		\end{equation}
		对于 \SI{1}{\mol} 物质,有
		\begin{equation} \label{EQ_FUNDAMENTAL_EQUATION_WITH_CHEMICAL_POTENTIAL}
			\dd \m = \dd G_\text{m} = -S_\text{m} \dd T + V_\text{m} \dd p \comma
		\end{equation}
		这是关于 $\m$ 的基本微分方程。
		
		\blankline
		
		定义\emphA{巨势(grand potential)}或\emphA{巨热力学势}
		\begin{equation} \label{EQ_DEF_OF_GRAND_POTENTIAL}
			\Y \eqdef F -\m n = F - G = -p \dd V \comma
		\end{equation}
		则其微分
		\begin{align}
			\dd \Y &= \dd F - \dd \; (\m n) \notag \\
			&= -S \dd T - p \dd V + \m \dd n - (\m \dd n + n \dd \mu) \notag \\
			&= -S \dd T - p \dd V - n \dd \mu \fullstop \footnotemark
		\end{align} \footnotetext{根据式~\eqref{EQ_DEF_OF_GRAND_POTENTIAL},可知 $\dd \Y = -\dd \; (p V)$。实际上,
		\begin{align*}
			-S \dd T - p \dd V - n \dd \mu &= -S \dd T - p \dd V - n (-S_\text{m} \dd T + V_\text{m} \dd p) \\
			&= -S \dd T - p \dd V + S \dd T - V \dd p = -\dd \; (p V) \comma
		\end{align*}
		这是不矛盾的。}
		在统计物理中,巨势与巨配分函数有关。
		
\section{平衡条件与稳定条件} \label{SEC_平衡条件与稳定条件}
	\subsection{平衡条件} \label{SUBSEC_平衡条件}
		对于一个\emphB{单元两相}(即一种组分,两种状态)的系统,其平衡条件为
		\begin{equation}
			\vd S = \vd S_1 + \vd S_2 = 0 \comma
		\end{equation}
		其中的下标“1”和“2”分别表示两个相。根据热力学基本方程,有
		\begin{braceEq}
			\vd S_1 &= \frac{1}{T_1} (\vd U_1 + p_1 \vd V_1 - \m_1 \vd n_1) \comma \\
			\vd S_2 &= \frac{1}{T_2} (\vd U_2 + p_2 \vd V_2 - \m_2 \vd n_2) \fullstop
		\end{braceEq}
		设整个系统是孤立的,则
		\begin{braceEq}
			\vd U_1 + \vd U_2 &= 0 \comma \\
			\vd V_1 + \vd V_2 &= 0 \comma \\
			\vd n_1 + \vd n_2 &= 0 \fullstop
		\end{braceEq}
		因此
		\begin{align}
			\vd S &= \vd S_1 + \vd S_2 \notag \\
			&= \left[ \frac{1}{T_1} (\vd U_1 + p_1 \vd V_1 - \m_1 \vd n_1) \right] + \left[ \frac{1}{T_2} (\vd U_2 + p_2 \vd V_2 - \m_2 \vd n_2) \right] \notag \\
			&= \left[ \frac{1}{T_1} (\vd U_1 + p_1 \vd V_1 - \m_1 \vd n_1) \right] + \left[ \frac{1}{T_2} (-\vd U_2 - p_2 \vd V_2 + \m_2 \vd n_2) \right] \notag \\
			&= \vd U_1 \left( \frac{1}{T_1} - \frac{1}{T_2} \right) + \vd V_1 \left( \frac{p_1}{T_1} - \frac{p_2}{T_2} \right) - \vd n_1 \left( \frac{\m_1}{T_1} - \frac{\m_2}{T_2} \right) = 0 \fullstop
		\end{align}
		由于 $U_1$、$V_1$ 和 $n_1$ 是相互独立的,所以它们的系数都应该等于零,即
		\begin{braceEq}
			T_1 &= T_2 \comma \quad \text{(热平衡条件)} \\
			p_1 &= p_2 \comma \quad \text{(力学平衡条件)} \\
			\m_1 &= \m_2 \fullstop \quad \text{(化学平衡条件)}%TODO:20160420 相变平衡条件
		\end{braceEq}
		若 $T$ 不等,则仍有能量流动;$p$ 不等,则两相界面可以移动;$\m$ 不等,则每一相中的粒子数仍在变化,因此都不是平衡。
		
	\subsection{稳定条件}
		计算熵的二级变分:
		\begin{align}
			\vd^{\:2} \! S_1 = \frac{1}{2} \left( \frac{\pd^{\:2} \! S_1}{\pd U_1^2} \right) %TODO:20160420 (δS)^2还是δ^2 S
		\end{align}
		
\section{单元系的复相平衡}%;相变分类;Ehrenfest公式}
	\subsection{单元系的相图}
		设单元系中有两个相($\a$ 和 $\b$),根据\secref{SEC_平衡条件与稳定条件}\subsecref{SUBSEC_平衡条件},其平衡条件为
		\begin{braceEq}
			T^\a &= T^\b \comma \\
			p^\a &= p^\b \comma \\
			\m^\a &= \m^\b \fullstop
		\end{braceEq}
		令 $T$、$p$ 分别为两相共同的温度和压强,并取它们为独立变量,则相变的平衡条件为
		\begin{equation} \label{EQ_PHASE_TRANSFORMATION_CONDITION_FOR_1_COMPOSITION_AND_2_PHASE}
			\m^\a (T, \, p) = \m^\b (T, \, p) \fullstop
		\end{equation}
		这实际上就是 $p$-$T$ 平面内一条曲线的方程。
		
		如果存在三个相($\a$、$\b$ 和 $\g$)并且同时达到平衡,则有
		\begin{equation}
			\m^\a (T, \, p) = \m^\b (T, \, p) = \m^\g (T, \,p) \fullstop
		\end{equation}
		这表示 $p$-$T$ 平面内的一个点,即\emphA{三相点}。
		
		\begin{figure}[h]
			\centering
%			\includegraphics[width = 6 cm]{Phase_diagram_of_water.svg}
			\caption{水的相图}
			\label{FIG_PHASE_DIAGRAM_OF_WATER}
		\end{figure}
		
		单元系的平衡性质利用\emphA{相图}可以清楚地表示。图~\ref{FIG_PHASE_DIAGRAM_OF_WATER} 显示的是水的相图。在三相点处,水的固、液、气三相共存。该点对应的温度为 \SI{273.16}{\kelvin},压强为 \SI{611.657}{\pascal}。另一个值得注意的点是\emphA{临界点},气液两相的平衡曲线终结于此处。对于水,该点对应的温度为 \SI{647}{\kelvin},压强为 \SI{22.064}{\mega\pascal}。
		
	\subsection{Clausius-Clapeyron方程}
		由式~\eqref{EQ_PHASE_TRANSFORMATION_CONDITION_FOR_1_COMPOSITION_AND_2_PHASE},单元系中两相平衡的条件为
		\begin{equation}
			\m^\a (T, \, p) = \m^\b (T, \, p) \fullstop
		\end{equation}
		设 $(T + \dd T, \, p + \dd p)$ 为平衡曲线上邻近 $(T, \, p)$ 的一点。当 $T$ 和 $p$ 沿平衡曲线变化到 $(T + \dd T, \, p + \dd p)$ 时,$\m^\a$ 和 $\m^\b$ 仍将满足平衡条件,即
		\begin{equation}
			\m^\a (T + \dd T, \, p + \dd p) = \m^\b (T + \dd T, \, p + \dd p) \fullstop
		\end{equation}
		两边都在点 $(T, \, p)$ 处作Taylor展开,可得
		\begin{equation}
			\m^\a (T, \, p) + \dd \m^\a = \m^\b (T, \, p) + \dd \m^\b \fullstop
		\end{equation}
		因此有
		\begin{equation} \label{EQ_dμα=dμβ}
			\dd \m^\a = \dd \m^\b \fullstop
		\end{equation}
		根据关于 $\m$ 的基本微分方程 \eqref{EQ_FUNDAMENTAL_EQUATION_WITH_CHEMICAL_POTENTIAL} 式,可知
		\begin{braceEq}
			\dd \m^\a &= -S^\a_\text{m} \dd T + V^\a_\text{m} \dd p \comma \\
			\dd \m^\b &= -S^\b_\text{m} \dd T + V^\b_\text{m} \dd p \fullstop
		\end{braceEq}
		将其代入式~\eqref{EQ_dμα=dμβ},便有
		\begin{equation} \label{EQ_CLAUSIUS_CLAPEYRON_RELATION_1}
			\frac{\dd p}{\dd T} = \frac{\displaystyle S^\a_\text{m} - S^\b_\text{m}}{\displaystyle V^\a_\text{m} - V^\b_\text{m}} \fullstop
		\end{equation}
		
		定义\emphA{相变潜热}
		\begin{equation}
			L_{\a \b} = T \left( S^\a_\text{m} - S^\b_\text{m} \right) \comma
		\end{equation}
		它表示 \SI{1}{\mol} 物质在保持温度、压强不变的条件下由 $\a$ 相转变为 $\b$ 相所吸收的热量。这样,\eqref{EQ_CLAUSIUS_CLAPEYRON_RELATION_1} 式就可以表述为
		\begin{equation}
			\frac{\dd p}{\dd T} = \frac{L_{\a \b}}{\displaystyle T \left( V^\a_\text{m} - V^\b_\text{m} \right)} \comma
		\end{equation}
		此即\emphA{Clausius–Clapeyron方程}。
		
		通常情况下,液体到气体的相变满足 $\dd p / \dd T > 0$;而液体到固体的相变满足 $\dd p / \dd T < 0$。需注意,水是一个常见的例外。液态的水变成冰,$\dd p / \dd T > 0$。
		
	\subsection{相变的分类}
		根据Ehrenfest的分类方案,$n$ 级相变对应Gibbs函数 $n$ 阶导数的不连续变化。相变点处,化学势连续,而化学势一阶导数不连续的相变称为\emphA{一级相变};化学势及其一阶导数均连续,而二阶导数不连续的相变称为\emphA{二级相变}。
	\subsection{Ehrenfest方程}
\section{气液相变;临界点行为}
\section{临界指数;普适性}
\section{Landau平均场理论}
	
\raggedbottom%FIXME:20260325 交叉引用、脚注每页重新计数失效,必须加上该行
\pagebreak
			
%		\chapter{多元复相 热力学第三定律}
%			\section{多元系热力学函数及热力学方程}
\section{多元系的复相平衡}
\section{Gibbs相律}
\section{混合理想气体}
\section{热力学第三定律}

%			
%	\part{统计物理}
%		\chapter{统计物理学基本概念}
%			\section{微观态的经典及量子描述}
%\section{}
%\section{}
%\section{}

		
%		\chapter{统计}
%			\include{Chapters/Chapter6}
%		
%		\chapter{系综理论}
%			\include{Chapters/Chapter7}
%		
%		\chapter{相变的统计物理简介}
%			\include{Chapters/Chapter8}
		
\end{document}