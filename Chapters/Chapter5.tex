\section{微观态的经典及量子描述}
	\subsection{单粒子的经典描述}
		微观态的经典描述以经典力学为基础,通常采用广义坐标与广义动量的形式。
		
		对于一个有 $r$ 个自由度的系统,需要用 $2r$ 个变量来描述其运动状态,即 $r$ 个广义坐标和 $r$ 个广义动量:
		\begin{equation}
			(q_i, \, p_i) \quad (i = 1, \, 2, \, \cdots, r) \fullstop
		\end{equation}
		系统的Hamilton量为
		\begin{equation}
			H = H(q_1, \, q_2, \, \cdots, q_r; \, p_1, \, p_2, \, \cdots, p_r; \, t) \comma
		\end{equation}
		正则方程为
		\begin{braceEq}
			\dot{q_i} &= \frac{\pd H}{\pd p_i} \comma \\
			\dot{p_i} &= -\frac{\pd H}{\pd q_i} \fullstop
		\end{braceEq}
		
		坐标和动量 $(q_1, \, q_2, \, \cdots, q_r; \, p_1, \, p_2, \, \cdots, p_r)$ 张成了一个 $2r$ 维空间,称为\emphA{\itshape{μ}空间},每一组坐标和动量描述的点称为\emphA{代表点}。
		
		\begin{myExample}[自由粒子]
			对于一个 $r = 3$ 的自由粒子,有
			\begin{braceEq}
				p_x = m \dot{x} \comma \\
				p_y = m \dot{y} \comma \\
				p_z = m \dot{z} \fullstop
			\end{braceEq}
			其 $\m$ 空间由 $(x, \, y, \, z; \, p_x, \, p_y, \, p_z)$ 张成。Hamilton量
			\begin{equation}
				H = \frac{1}{2m} \left( p_x^2 + p_y^2 +p_z^2 \right) \fullstop
			\end{equation}
			%TODO:20160504 示意图
		\end{myExample}
		
		\begin{myExample}[一维谐振子]
			质量为 $m$ 的物体受力 $F = -A x$ 的作用做简谐运动,其角频率
			\begin{equation}
				\o = \sqrt{\frac{A}{m}} \comma
			\end{equation}
			则其Hamilton量
			\begin{equation}
				H = \frac{p^2}{2m} + \frac{A}{2} x^2 = \frac{p^2}{2m} + \frac{1}{2} m \o^2 x^2 \fullstop
			\end{equation}
			
			若总能量一定,即 $H = E$,则
			\begin{equation}
				\frac{p^2}{2mE} + \frac{x^2}{2E / m \o^2} = 1 \comma
			\end{equation}
			这在 $\m$ 空间中表示一个椭圆(见图),其面积
			\begin{equation}
				S_\text{椭圆} = \p \, a b = \p \cdot \sqrt{\frac{2E}{m \o^2}} \cdot \sqrt{2mE} = \frac{2 \p E}{\o} \fullstop
			\end{equation}
			%TODO:20160504 示意图
		\end{myExample}
		
		\begin{myExample}[转子]
			%TODO:20160624 未完成
		\end{myExample}
		
	\subsection{单粒子的量子描述}
		微观态的量子描述以量子力学为基础。粒子的动量 $\vecb{p}$、能量 $E$ 满足\emphA{de Broglie关系}:
		\begin{braceEq}
			\vecb{p} &= \hbar \vecb{k} \comma \\
			E &= \hbar \o \comma
		\end{braceEq}
		其中的 $\hbar$ 称为\emphA{(约化)Planck常数},其值为
		\begin{equation}
			\hbar = \frac{h}{2\p} = \SI{1.0545718e-34}{\joule\second} \fullstop
		\end{equation}
		式中的 $h$ 也是Planck常数。
		
		de Broglie关系说明微观粒子具有\emphA{波粒二象性}。这就引出了另一个重要结果——\emphA{不确定关系}\footnote{
			更精确的表述为 $\incr p \incr q \geqslant \hbar / 2 $。
		}:
		\begin{equation}
			\incr p \incr q \gtrsim h \fullstop
		\end{equation}
		由此,粒子的动量和坐标不可能被同时精确测量,因而其运动也就无法用经典的轨道概念描述,必须改用波函数来描述。
		
		粒子波函数 $\Y$ 满足的方程即\emphA{Schrödinger方程}:
		\begin{equation}
			\ii \frac{\pd}{\pd t} \Y = \op{H} \Y \comma
		\end{equation}
		式中的 $\op{H}$ 是Hamilton算符。在定态情况(即将时间变量分离后),Schrödinger方程化为
		\begin{equation}
			\op{H} \Y = E \Y \fullstop
		\end{equation}
		
		\begin{myExample}[立方体容器内的自由粒子]
			设粒子在边长为 $L$ 的立方体容器内运动,则其量子态(即波函数)有平面波的形式:
			\begin{equation}
				\Y_{n_1, \, n_2, \, n_3} (\vecb{r}) \sim \ee^{\ii \vecb{p} \cdot \vecb{r} / \hbar} \comma
				\quad n_i = \pm 1, \, \pm 2, \, \pm 3, \, \cdots \fullstop
			\end{equation}
			求解Schrödinger方程,可以发现动量与能量的本征值都是量子化的:
			\begin{braceEq}
				\vecb{p} &= p_x \vecb{i} + p_y \vecb{j} + p_z \vecb{k}
				= \frac{2\p\hbar}{L} n_1 \vecb{i} + n_2 \vecb{j} + n_3 \vecb{k} \comma \\
				E &= \frac{1}{2m} \left( p_x^2+p_y^2+p_z^2 \right)
				= \frac{\displaystyle 2\p^2 \hbar^2}{\displaystyle mL^2} \left( n_1^2+n_2^2+n_3^2 \right) \fullstop
			\end{braceEq} %HACK:20160624 乘方位置调整
			量子化的能量也成为\emphA{能级}。对于能级
			\begin{equation}
				E = \frac{\displaystyle 2\p^2 \hbar^2}{\displaystyle mL^2} \comma
			\end{equation}
			它对应6种量子态:
			\begin{equation}
				\left(\pm 1, \, 0, \, 0 \right),\,
				\left(0, \, \pm 1, \, 0 \right),\,
				\left(0, \, 0, \, \pm 1 \right) \fullstop
			\end{equation}
			这种现象称为能级\emphA{简并}。同一能级对应量子态的数目称为\emphA{简并度}。显然,这里的简并度为6。而能量更高的一个能级
			\begin{equation}
				E = \frac{\displaystyle 2\p^2 \hbar^2}{\displaystyle mL^2} \times 3
				=\frac{\displaystyle 2\p^2 \hbar^2}{\displaystyle mL^2} \times (1+1+1)
			\end{equation}
			则对应 $2^3=8$ 个量子态,它的简并度为8。
		\end{myExample}
		
		\begin{myExample}[一维谐振子]
			
		\end{myExample}
		
		\begin{myExample}[转子]
			
		\end{myExample}%TODO:20160624 未完成
		
	\subsection{多粒子体系的描述}
		对于 $N$ 个粒子组成的体系,设每个粒子的自由度为 $r$,则每个粒子可用 $2r$ 个变量 $(q_1, \, q_2, \, \cdots, q_r; \allowbreak \, p_1, \, p_2, \, \cdots, p_r)$ 来描述。此时,系统的总自由度 %FIXME:20160624 公式换行
		\begin{equation}
			f = Nr \comma
		\end{equation}
		因而系统的运动需要用 $2f$ 个变量来刻画。它们张成了一个 $2f$ 维空间,称为\emphA{\itshape{Γ}空间},也叫\emphA{相空间}。
%\section{}
%\section{}
%\section{}

\raggedbottom%FIXME:20260325 交叉引用、脚注每页重新计数失效,必须加上该行
\pagebreak
