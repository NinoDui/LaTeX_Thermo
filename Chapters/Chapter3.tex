我们所研究的系统是逐渐复杂的。首先是\emphB{独立子体系},如理想气体、Fermi气体、Bose气体等;之后是\emphB{近独立子体系},如准粒子气体;%TODO:20160413 翻译
本章主要探讨\emphB{相互作用体系},它包含多种物态,也会有相变的发生。

\section{热动平衡判据;开系热力学}
	在 \secref{SEC_熵增加原理与最大功}中我们已经证明,对于孤立体系(即绝热过程),熵将不断增加,直至达到极大(平衡)。因此,熵达到极大便可以作为体系达到平衡的判据。
	
	\subsection{平衡判据}
		\begin{myEnum1}
			\myItem{熵判据}
				由上,在 $\vd U = 0, \, \vd V = 0, \, \vd N = 0$ 的前提下,$\vd S = 0, \, \vd^{\:2} \! S <0$ 即说明(孤立)体系达到了平衡。这里的“$\vd\,$”表示\emphB{虚变化},与“$\dd\,$”代表的\emphB{真实变化}有所不同,它与虚功原理中的虚位移是类似的。%HACK:20160420 δS的缩进
				
				平衡分为三种:稳定平衡、亚稳平衡和不稳定平衡。如前所述,熵的极大值对应平衡态。稳定平衡对应其中\emphB{最大}的极大值,而亚稳平衡对应其他较小的极大值,即对于无限小的变动是稳定的,而对于有限的变动则是不稳定的。不稳定平衡对应极小值,虽然有 $\vd S = 0$,但却不满足 $\vd^{\:2} \! S <0$。平衡的稳定性可以用力学类比来理解。重力势能 $E_\text{p}$ 的极小值对应平衡。其最小值对应稳定平衡;但相对极小对于大的扰动是不稳定的,所以对应亚稳平衡。而极大值则对应不稳定平衡,稍有扰动就会偏离。%TODO:20160420 图片
				
				对于 $\vd S = 0$、$\vd^{\:2} \! S =0$ 的临界态,将 $S$ 围绕极值点Taylor作展开,可得
				\begin{equation}
					\tl{\incr\,} S = \vd S + \frac{1}{2!} \vd^{\:2} \! S + \frac{1}{3!} \vd^{\:3} \! S + \frac{1}{4!} \vd^{\:4} \! S + \cdots < 0 \fullstop
				\end{equation}
				式中的“$\tl{\incr\,}$”同样表示虚变动。由于要满足 $S \rightarrow -S$ 的对称性,因此 $\vd^{\:3} \! S = 0$。从而%TODO:20160420 三阶项为何等于零
				\begin{equation}
					\vd^{\:4} \! S < 0 \comma
				\end{equation}
				这就是临界态的平衡判据。
				
			\myItem{自由能判据}
				考虑一个系统与热库(即环境)组成的复合体系,它是一个孤立系。显然,总内能
				\begin{equation}
					U_0 = U_\text{sys} + U_\text{res} = \const \comma
				\end{equation}
				总体积
				\begin{equation}
					V_0 = V_\text{sys} + V_\text{res} = \const
				\end{equation}
				设想体系发生了一个虚变动,则
				\begin{braceEq}[gather]
					\vd U_\text{sys} + \vd U_\text{res} = 0 \comma \\
					\vd V_\text{sys} + \vd V_\text{res} = 0 \fullstop \label{EQ_DELTA_V_SYS+DELTA_V_RES=0}
				\end{braceEq}
				根据熵判据,
				\begin{braceEq}
					\vd S_0 &= \vd \; (S_\text{sys} + S_\text{res}) = 0 \comma \\
					\vd^{\:2} \! S_0 &= \vd^{\:2} (S_\text{sys} + S_\text{res}) < 0 \fullstop
				\end{braceEq}
				根据热力学基本微分方程 \eqref{EQ_FUNDAMENTAL_EQUATION_FOR_PVT_SYSTEM} 式,可得
				\begin{equation}
					\vd U_\text{res} = T_\text{res} \vd S_\text{res} + p_\text{res}  \vd V_\text{res} \fullstop \footnote{
						已经把式~\eqref{EQ_FUNDAMENTAL_EQUATION_FOR_PVT_SYSTEM} 中的微分“$\dd\,$”改成了变分“$\vd\,$”的形式。
					}
				\end{equation}
				
				在系统温度、体积均不变(即 $\vd T_\text{sys} = 0$、$\vd V_\text{sys} = 0$)的情形下,根据式~\eqref{EQ_DELTA_V_SYS+DELTA_V_RES=0} 可知 $\vd V_\text{res} = 0$。由于是平衡态,又有 $T_\text{res} = T_\text{sys}$。因此
				
				\begin{align}
					\vd F_\text{sys} &= \vd \; (U_\text{sys} - T_\text{sys} S_\text{sys}) \\
					&= \vd U_\text{sys} - T_\text{sys} \vd S_\text{sys} \\
					&= -\vd U_\text{res} + T_\text{res} \vd S_\text{res} \\
					&= -p_\text{res}  \vd V_\text{res} \\
					&= -p_\text{res}  \vd V_\text{sys} = 0 \comma \\
					\vd^{\:2} \! F_\text{sys} &= %TODO:20160420 推导过程
				\end{align}
				
			\myItem{Gibbs函数判据}
			\myItem{内能判据}
		\end{myEnum1}
		
	\subsection{开系热力学}
		所谓“开系”,即\emphA{开放系(open system)},它指粒子数可变且有能量交换的系统。
		
		对于封闭系,我们已经知道
		\begin{equation}
			\dd G = -S \dd T + V \dd p \fullstop
		\end{equation}
		而对于开放系,需引进\emphA{化学势} $\m$,其定义为
		\begin{equation}
			\m \eqdef \myPartial{G}{n}{T, \, p} \fullstop \footnote{这里的$n$指物质的量,而非粒子数。定义$\m \eqdef (\pd G / \pd N)_{T,\,p}$ 当然也可以,不过我们只采用第一种定义。}
		\end{equation}
		因此,
		\begin{equation}
			\dd G = -S \dd T + V \dd p + \m \dd n \fullstop
		\end{equation}
		利用Legendre变换,可得粒子数可变系统的热力学基本方程:
		\begin{braceEq}
			\dd U &= T \dd S - p \dd V + \m \dd n \comma \\
			\dd H &= T \dd S + V \dd p + \m \dd n \comma \\
			\dd F &= -S \dd T - p \dd V + \m \dd n \fullstop
		\end{braceEq}
		于是可以得出 $\m$ 的几个等价定义:
		\begin{equation}
			\m = \myPartial{U}{n}{S, \, V} = \myPartial{H}{n}{S, \, p} = \myPartial{F}{n}{T, \, V} = \myPartial{G}{n}{T, \, p} \fullstop
		\end{equation}
		
		Gibbs函数 $G(T,\,p,\,n)$ 是广延量。因此可定义\emphA{摩尔Gibbs函数} $G_\text{m}$,使得
		\begin{equation}
			G(T,\,p,\,n) = n G_\text{m} (T,\,p) \fullstop
		\end{equation}
		因此,
		\begin{equation}
			\m \eqdef \myPartial{G}{n}{T, \, p} = \left[ \frac{\pd}{\pd n} (n G_\text{m}) \right]_{T, \, p} = G_\text{m} \fullstop
		\end{equation}
		对于 \SI{1}{\mol} 物质,有
		\begin{equation} \label{EQ_FUNDAMENTAL_EQUATION_WITH_CHEMICAL_POTENTIAL}
			\dd \m = \dd G_\text{m} = -S_\text{m} \dd T + V_\text{m} \dd p \comma
		\end{equation}
		这是关于 $\m$ 的基本微分方程。
		
		\blankline
		
		定义\emphA{巨势(grand potential)}或\emphA{巨热力学势}
		\begin{equation} \label{EQ_DEF_OF_GRAND_POTENTIAL}
			\Y \eqdef F -\m n = F - G = -p \dd V \comma
		\end{equation}
		则其微分
		\begin{align}
			\dd \Y &= \dd F - \dd \; (\m n) \notag \\
			&= -S \dd T - p \dd V + \m \dd n - (\m \dd n + n \dd \mu) \notag \\
			&= -S \dd T - p \dd V - n \dd \mu \fullstop \footnotemark
		\end{align} \footnotetext{根据式~\eqref{EQ_DEF_OF_GRAND_POTENTIAL},可知 $\dd \Y = -\dd \; (p V)$。实际上,
		\begin{align*}
			-S \dd T - p \dd V - n \dd \mu &= -S \dd T - p \dd V - n (-S_\text{m} \dd T + V_\text{m} \dd p) \\
			&= -S \dd T - p \dd V + S \dd T - V \dd p = -\dd \; (p V) \comma
		\end{align*}
		这是不矛盾的。}
		在统计物理中,巨势与巨配分函数有关。
		
\section{平衡条件与稳定条件} \label{SEC_平衡条件与稳定条件}
	\subsection{平衡条件} \label{SUBSEC_平衡条件}
		对于一个\emphB{单元两相}(即一种组分,两种状态)的系统,其平衡条件为
		\begin{equation}
			\vd S = \vd S_1 + \vd S_2 = 0 \comma
		\end{equation}
		其中的下标“1”和“2”分别表示两个相。根据热力学基本方程,有
		\begin{braceEq}
			\vd S_1 &= \frac{1}{T_1} (\vd U_1 + p_1 \vd V_1 - \m_1 \vd n_1) \comma \\
			\vd S_2 &= \frac{1}{T_2} (\vd U_2 + p_2 \vd V_2 - \m_2 \vd n_2) \fullstop
		\end{braceEq}
		设整个系统是孤立的,则
		\begin{braceEq}
			\vd U_1 + \vd U_2 &= 0 \comma \\
			\vd V_1 + \vd V_2 &= 0 \comma \\
			\vd n_1 + \vd n_2 &= 0 \fullstop
		\end{braceEq}
		因此
		\begin{align}
			\vd S &= \vd S_1 + \vd S_2 \notag \\
			&= \left[ \frac{1}{T_1} (\vd U_1 + p_1 \vd V_1 - \m_1 \vd n_1) \right] + \left[ \frac{1}{T_2} (\vd U_2 + p_2 \vd V_2 - \m_2 \vd n_2) \right] \notag \\
			&= \left[ \frac{1}{T_1} (\vd U_1 + p_1 \vd V_1 - \m_1 \vd n_1) \right] + \left[ \frac{1}{T_2} (-\vd U_2 - p_2 \vd V_2 + \m_2 \vd n_2) \right] \notag \\
			&= \vd U_1 \left( \frac{1}{T_1} - \frac{1}{T_2} \right) + \vd V_1 \left( \frac{p_1}{T_1} - \frac{p_2}{T_2} \right) - \vd n_1 \left( \frac{\m_1}{T_1} - \frac{\m_2}{T_2} \right) = 0 \fullstop
		\end{align}
		由于 $U_1$、$V_1$ 和 $n_1$ 是相互独立的,所以它们的系数都应该等于零,即
		\begin{braceEq}
			T_1 &= T_2 \comma \quad \text{(热平衡条件)} \\
			p_1 &= p_2 \comma \quad \text{(力学平衡条件)} \\
			\m_1 &= \m_2 \fullstop \quad \text{(化学平衡条件)}%TODO:20160420 相变平衡条件
		\end{braceEq}
		若 $T$ 不等,则仍有能量流动;$p$ 不等,则两相界面可以移动;$\m$ 不等,则每一相中的粒子数仍在变化,因此都不是平衡。
		
	\subsection{稳定条件}
		计算熵的二级变分:
		\begin{align}
			\vd^{\:2} \! S_1 = \frac{1}{2} \left( \frac{\pd^{\:2} \! S_1}{\pd U_1^2} \right) %TODO:20160420 (δS)^2还是δ^2 S
		\end{align}
		
\section{单元系的复相平衡}%;相变分类;Ehrenfest公式}
	\subsection{单元系的相图}
		设单元系中有两个相($\a$ 和 $\b$),根据\secref{SEC_平衡条件与稳定条件}\subsecref{SUBSEC_平衡条件},其平衡条件为
		\begin{braceEq}
			T^\a &= T^\b \comma \\
			p^\a &= p^\b \comma \\
			\m^\a &= \m^\b \fullstop
		\end{braceEq}
		令 $T$、$p$ 分别为两相共同的温度和压强,并取它们为独立变量,则相变的平衡条件为
		\begin{equation} \label{EQ_PHASE_TRANSFORMATION_CONDITION_FOR_1_COMPOSITION_AND_2_PHASE}
			\m^\a (T, \, p) = \m^\b (T, \, p) \fullstop
		\end{equation}
		这实际上就是 $p$-$T$ 平面内一条曲线的方程。
		
		如果存在三个相($\a$、$\b$ 和 $\g$)并且同时达到平衡,则有
		\begin{equation}
			\m^\a (T, \, p) = \m^\b (T, \, p) = \m^\g (T, \,p) \fullstop
		\end{equation}
		这表示 $p$-$T$ 平面内的一个点,即\emphA{三相点}。
		
		\begin{figure}[h]
			\centering
%			\includegraphics[width = 6 cm]{Phase_diagram_of_water.svg}
			\caption{水的相图}
			\label{FIG_PHASE_DIAGRAM_OF_WATER}
		\end{figure}
		
		单元系的平衡性质利用\emphA{相图}可以清楚地表示。图~\ref{FIG_PHASE_DIAGRAM_OF_WATER} 显示的是水的相图。在三相点处,水的固、液、气三相共存。该点对应的温度为 \SI{273.16}{\kelvin},压强为 \SI{611.657}{\pascal}。另一个值得注意的点是\emphA{临界点},气液两相的平衡曲线终结于此处。对于水,该点对应的温度为 \SI{647}{\kelvin},压强为 \SI{22.064}{\mega\pascal}。
		
	\subsection{Clausius-Clapeyron方程}
		由式~\eqref{EQ_PHASE_TRANSFORMATION_CONDITION_FOR_1_COMPOSITION_AND_2_PHASE},单元系中两相平衡的条件为
		\begin{equation}
			\m^\a (T, \, p) = \m^\b (T, \, p) \fullstop
		\end{equation}
		设 $(T + \dd T, \, p + \dd p)$ 为平衡曲线上邻近 $(T, \, p)$ 的一点。当 $T$ 和 $p$ 沿平衡曲线变化到 $(T + \dd T, \, p + \dd p)$ 时,$\m^\a$ 和 $\m^\b$ 仍将满足平衡条件,即
		\begin{equation}
			\m^\a (T + \dd T, \, p + \dd p) = \m^\b (T + \dd T, \, p + \dd p) \fullstop
		\end{equation}
		两边都在点 $(T, \, p)$ 处作Taylor展开,可得
		\begin{equation}
			\m^\a (T, \, p) + \dd \m^\a = \m^\b (T, \, p) + \dd \m^\b \fullstop
		\end{equation}
		因此有
		\begin{equation} \label{EQ_dμα=dμβ}
			\dd \m^\a = \dd \m^\b \fullstop
		\end{equation}
		根据关于 $\m$ 的基本微分方程 \eqref{EQ_FUNDAMENTAL_EQUATION_WITH_CHEMICAL_POTENTIAL} 式,可知
		\begin{braceEq}
			\dd \m^\a &= -S^\a_\text{m} \dd T + V^\a_\text{m} \dd p \comma \\
			\dd \m^\b &= -S^\b_\text{m} \dd T + V^\b_\text{m} \dd p \fullstop
		\end{braceEq}
		将其代入式~\eqref{EQ_dμα=dμβ},便有
		\begin{equation} \label{EQ_CLAUSIUS_CLAPEYRON_RELATION_1}
			\frac{\dd p}{\dd T} = \frac{\displaystyle S^\a_\text{m} - S^\b_\text{m}}{\displaystyle V^\a_\text{m} - V^\b_\text{m}} \fullstop
		\end{equation}
		
		定义\emphA{相变潜热}
		\begin{equation}
			L_{\a \b} = T \left( S^\a_\text{m} - S^\b_\text{m} \right) \comma
		\end{equation}
		它表示 \SI{1}{\mol} 物质在保持温度、压强不变的条件下由 $\a$ 相转变为 $\b$ 相所吸收的热量。这样,\eqref{EQ_CLAUSIUS_CLAPEYRON_RELATION_1} 式就可以表述为
		\begin{equation}
			\frac{\dd p}{\dd T} = \frac{L_{\a \b}}{\displaystyle T \left( V^\a_\text{m} - V^\b_\text{m} \right)} \comma
		\end{equation}
		此即\emphA{Clausius–Clapeyron方程}。
		
		通常情况下,液体到气体的相变满足 $\dd p / \dd T > 0$;而液体到固体的相变满足 $\dd p / \dd T < 0$。需注意,水是一个常见的例外。液态的水变成冰,$\dd p / \dd T > 0$。
		
	\subsection{相变的分类}
		根据Ehrenfest的分类方案,$n$ 级相变对应Gibbs函数 $n$ 阶导数的不连续变化。相变点处,化学势连续,而化学势一阶导数不连续的相变称为\emphA{一级相变};化学势及其一阶导数均连续,而二阶导数不连续的相变称为\emphA{二级相变}。
	\subsection{Ehrenfest方程}
\section{气液相变;临界点行为}
\section{临界指数;普适性}
\section{Landau平均场理论}
	
\raggedbottom%FIXME:20260325 交叉引用、脚注每页重新计数失效,必须加上该行
\pagebreak