我们所研究的系统是逐渐复杂的。首先是\emphB{独立子体系},如理想气体、Fermi气体、Bose气体等;之后是\emphB{近独立子体系},如准粒子气体;%TODO:20160413 翻译
本章主要探讨\emphB{相互作用体系},它包含多种物态,也会有相变的发生。

\section{热动平衡判据;开系热力学}
	在 \secref{SEC_熵增加原理与最大功}中我们已经证明,对于孤立体系(即绝热过程),熵将不断增加,直至达到极大(平衡)。因此,熵达到极大便可以作为体系达到平衡的判据。
	
	\subsection{平衡判据}
		\begin{myEnum1}
			\myItem{熵判据}
				由上所述,在 $\vd U = 0, \, \vd V = 0, \, \vd N = 0$ 的前提下,$\vd S = 0, \, \vd^{\:2} \! S <0$ 即说明(孤立)体系达到了平衡。这里的“$\vd\,$”表示\emphB{虚变化},与“$\dd\,$”代表的真实变化有所不同,它与虚功原理中的虚位移是类似的。
				
				平衡分为三种:稳定平衡、亚稳平衡和不稳定平衡。稳定平衡对应
			\myItem{自由能判据}
			\myItem{Gibbs函数判据}
			\myItem{内能判据}
		\end{myEnum1}
		
	\subsection{开系热力学}
		所谓“开系”,即\emphA{开放系(open system)},它指粒子数可变且有能量交换的系统。
		
		对于封闭系,我们已经知道
		\begin{equation}
			\dd G = -S \dd T + V \dd p \fullstop
		\end{equation}
		而对于开放系,需引进\emphA{化学势} $\m$,其定义为
		\begin{equation}
			\m \eqdef \myPartial{G}{N}{T,\,p} \fullstop
		\end{equation}
		因此,
		\begin{equation}
			\dd G = -S \dd T + V \dd p + \m \dd N \fullstop
		\end{equation}%TODO:20160413 n还是N
	
\section{平衡条件;稳定性判据}
\section{单元系的复相平衡;相变分类;Ehrenfest公式}
\section{气液相变;临界点行为}
\section{临界指数;普适性}
\section{Landau平均场理论}
	
\raggedbottom%FIXME:20160315 交叉引用、脚注每页重新计数失效,必须加上该行
\pagebreak