\section{Maxwell关系} \label{SEC_Maxwell关系}
	\subsection{热力学函数的全微分}
		\begin{myEnum1}
			\myItem{$U$ 的全微分}
				根据热力学基本微分方程[见式~\eqref{EQ_FUNDAMENTAL_EQUATION_FOR_PVT_SYSTEM}]
				\begin{equation} \label{EQ_dU=TdS-pdV_IN_MAXWELL_RELATION}
					\dd U = T \dd S + p \dd V \comma
				\end{equation}
				可知内能 $U$ 是 $S$ 和 $V$ 的函数,即
				\begin{equation}
					U = U(S, \, V) \fullstop
				\end{equation}
				因此
				\begin{equation}
					\dd U = \myPartial{U}{S}{V} \dd S + \myPartial{U}{V}{S} \dd V \fullstop
				\end{equation}
				与式~\eqref{EQ_dU=TdS-pdV_IN_MAXWELL_RELATION} 进行比较,可得
				\begin{braceEq}
					& T = \myPartial{U}{S}{V} \comma  \label{EQ_T_IN_PARTIAL_DIFFERENTIAL_OF_U} \\
					& p = -\myPartial{U}{V}{S}  \label{EQ_p_IN_PARTIAL_DIFFERENTIAL_OF_U} \fullstop
				\end{braceEq}
				$U$ 的二阶偏导数与其先后次序无关,即
				\begin{equation}
					\frac{\pd^{\:2} \! U}{\pd V \pd S} = \frac{\pd^{\:2} \! U}{\pd S \pd V}%HACK:20160330 偏微分空格
				\end{equation}
				或
				\begin{equation}
					\left[ \frac{\pd}{\pd V} \myPartial{U}{S}{V}\right]_S
					= \left[ \frac{\pd}{\pd S} \myPartial{U}{V}{S}\right]_V \fullstop
				\end{equation}
				利用该式,对式~\eqref{EQ_T_IN_PARTIAL_DIFFERENTIAL_OF_U} 和式~\eqref{EQ_p_IN_PARTIAL_DIFFERENTIAL_OF_U} 两边分别求导,可得
				\begin{equation} \label{EQ_MAXWELL_RELATION_FROM_U}
					\myPartial{T}{V}{S} = -\myPartial{p}{S}{V} \fullstop
				\end{equation}
				
			\myItem{$H$ 的全微分}
				因为
				\begin{equation}
					H \eqdef U + p V \comma
				\end{equation}
				所以
				\begin{align}
					\dd H &= T \dd S - p \dd V + \dd \; (p V) \notag \\
					&= T \dd S - p \dd V + (V \dd p + p \dd V) \notag \\
					&= T \dd S + V \dd p \fullstop
				\end{align}%HACK:20160330 微分d后接空格,间距有问题
				这说明焓 $H$ 是 $S$ 和 $p$ 的函数,即
				\begin{equation}
					H = H(S, \, p) \fullstop
				\end{equation}
				因此
				\begin{equation}
					\dd H = \myPartial{H}{S}{p} \dd S + \myPartial{H}{p}{S} \dd p \fullstop
				\end{equation}
				进而
				\begin{braceEq}
					& T = \myPartial{H}{S}{p} \comma \\
					& V = \myPartial{H}{p}{S} \fullstop
				\end{braceEq}
				与上文类似,对它们两边分别求导,并利用混合偏导数定理,就得到
				\begin{equation} \label{EQ_MAXWELL_RELATION_FROM_H}
					\myPartial{T}{p}{S} = \myPartial{V}{S}{p} \fullstop
				\end{equation}
				
				\myItem{$F$ 的全微分}
				因为
				\begin{equation}
					F \eqdef U - T S \comma
				\end{equation}
				所以
				\begin{equation} \label{EQ_dF=-SdT-pdV_IN_MAXWELL_RELATION}
					\dd F = -S \dd T - p \dd V \fullstop
				\end{equation}
				这说明自由能 $F$ 是 $T$ 和 $V$ 的函数,即
				\begin{equation}
					F = F(T, \, V) \fullstop
				\end{equation}
				因此
				\begin{equation}
					\dd F = \myPartial{F}{T}{V} \dd T + \myPartial{F}{V}{T} \dd V \fullstop
				\end{equation}
				进而
				\begin{braceEq}
					& S = -\myPartial{F}{T}{V} \comma \\
					& p = -\myPartial{F}{V}{T} \fullstop
				\end{braceEq}
				两边分别求导,得
				\begin{equation} \label{EQ_MAXWELL_RELATION_FROM_F}
					\myPartial{S}{V}{T} = \myPartial{p}{T}{V} \fullstop
				\end{equation}
			
			\myItem{$G$ 的全微分}
				因为
				\begin{equation}
					G \eqdef U - T S + p V \comma
				\end{equation}
				所以
				\begin{equation}
				\dd G = -S \dd T + V \dd p \fullstop
				\end{equation}
				这说明Gibbs自由能是 $T$ 和 $p$ 的函数,即
				\begin{equation}
					G = G(T, \, p) \fullstop
				\end{equation}
				因此
				\begin{equation}
					\dd H = \myPartial{G}{G}{p} \dd T + \myPartial{G}{p}{T} \dd T \fullstop
				\end{equation}
				进而
				\begin{braceEq}
					& S = -\myPartial{G}{T}{p} \comma \\
					& V = \myPartial{G}{p}{T} \fullstop
				\end{braceEq}
				两边分别求导,得
				\begin{equation} \label{EQ_MAXWELL_RELATION_FROM_G}
				\myPartial{S}{p}{T} = -\myPartial{V}{T}{p} \fullstop
				\end{equation}
		\end{myEnum1}
		
		\blankline
		式~\eqref{EQ_MAXWELL_RELATION_FROM_U}、\eqref{EQ_MAXWELL_RELATION_FROM_H}、\eqref{EQ_MAXWELL_RELATION_FROM_F} 和 \eqref{EQ_MAXWELL_RELATION_FROM_G} 称为\emphA{Maxwell关系}。
		
	\subsection{Legendre变换}
		\emphA{Legendre变换}是指等式
		\begin{equation}
			x \dd y = \dd \; (x y) - y \dd x \comma
		\end{equation}
		它把变量从 $x$ 变为了 $y$。
		
		在经典力学中,Legendre变换被用来从Lagrange表述导出Hamilton表述:
		\begin{equation}
			\scL (q_\a, \, \dot{q_\a}) \quad \rightarrow \quad \scH (q_\a, \, p_\a) = \sum_\a \dot{q_\a} p_\a - \scL (q_\a, \, \dot{q_\a}) \fullstop
		\end{equation}
		
		对式~\eqref{EQ_dU=TdS-pdV_IN_MAXWELL_RELATION} 应用Legendre变换
		\begin{equation}
			T \dd S = \dd \; (T S) - S \dd T
		\end{equation}
		并移项,就得到
		\begin{equation}
			\dd U - \dd \; (T S) = \dd \; (U - T S)= \dd F = -S \dd T + p \dd V \fullstop
		\end{equation}
		这也就是用\emphB{自由能}表示的热力学基本方程,即式~\eqref{EQ_dF=-SdT-pdV_IN_MAXWELL_RELATION}。由此,利用偏微分关系稍做计算,就得到了Maxwell关系 \eqref{EQ_MAXWELL_RELATION_FROM_F} 式。用同样的手法,也可以得到其他几个Maxwell关系。
		
		\begin{myTable}[H]{MM>{$}l<{$}}{Maxwell关系}{TAB_MAXWELL_RELATION}
			\toprule
			\text{\emphA{基本微分方程}} & \text{\emphA{自然变量}} & \text{\emphA{Maxwell关系}} \\
			\midrule
			\dd U = T \dd S - p \dd V & (S, \, V) & \myPartialDisplay{T}{V}{S} = -\myPartialDisplay{p}{S}{V} \\
			\dd H = \dd \; (U + p V) = T \dd S + V \dd p & (S, \, p) & \myPartialDisplay{T}{p}{S} = \myPartialDisplay{V}{S}{p} \\
			\dd F = \dd \; (U - T S) = -S \dd T - p \dd V & (T, \, V) & \myPartialDisplay{S}{V}{T} = \myPartialDisplay{p}{T}{V} \\
			\dd G = \dd \; (U - T S + p V) = -S \dd T + V \dd p & (T, \, p) & \myPartialDisplay{S}{p}{T} = -\myPartialDisplay{V}{T}{p} \\
			\bottomrule
		\end{myTable}%FIXME:20160330 表格格式
		
		表~\ref{TAB_MAXWELL_RELATION} 总结了上文的推导结果。其中的“\emphA{自然变量}”指经Legendre变换后的自由变量,它们直接出现在热力学基本方程中。
		
		热力学变量(或函数)[见 \secref{SEC_平衡态及其描述} \subsecref{SUBSEC_平衡态的描述}]可分为两种:\emphA{可测量量}和\emphA{不可测量量}。物态方程中的 $p$、$V$、$T$,与物态方程有关的 $\a$、$\b$、$\k_T$,以及热容,都是可测量量;而 $U$、$S$、$H$、$F$、$G$ 等状态函数,以及 $(\pd S / \pd V)_T$、$(\pd H / \pd p)_T$ 等偏微分,都是不可测量量。Maxwell关系的作用,就是把不可测量量用可测量量来表示。
		
	\subsection{简单应用} \label{SUBSEC_简单应用_OF_MAXWELL关系}
		\begin{myExample} \label{EG_C_p-C_V}
			计算 $C_p - C_V$。\footnote{
				注意与式~\eqref{EQ_C_p-C_V_FOR_IDEAL_GAS} 对比,那里要求理想气体。
			}%TODO:20160330 式1.4.8有问题
			
			根据 \secref{SEC_热容与焓;理想气体的性质} \subsecref{SUBSEC_热容与焓} 中的推导,
			\begin{braceEq}
				C_p &= \myPartial{H}{T}{p} \comma \\
				C_V &= \myPartial{U}{T}{V} \fullstop
			\end{braceEq}
			利用 $\db Q = T \dd S$,可得%TODO:20160330 为什么dQ=TdS(可逆?)
			\begin{braceEq}
				C_p &= T \myPartial{S}{T}{p} \comma \label{EQ_C_p_IN_T_AND_S} \\
				C_V &= T \myPartial{S}{T}{V} \fullstop \label{EQ_C_V_IN_T_AND_S} 
			\end{braceEq}%TODO:20160330 推导过程
			把 $S = S(T, \, p)$ 看成复合函数的形式,即 $S[T, \, V(T, \, p)]$,因此
			\begin{equation}
				\myPartial{S}{T}{p} = \myPartial{S}{T}{V} + \myPartial{S}{V}{T} \myPartial{V}{T}{p} \fullstop
			\end{equation}
			于是
			\begin{align}
				C_p - C_V &= T \left[ \myPartial{S}{T}{p} - \myPartial{S}{T}{V} \right] \notag \\
				&= T \myPartial{S}{V}{T} \myPartial{V}{T}{p} \notag \\%FIXME:20160401 强调
				&= T \myPartial{p}{T}{V} \myPartial{V}{T}{p} \fullstop
				\myTagNumbering{Maxwell关系}  \label{EQ_C_p-C_V_WITH_MAXWELL_RELATION_PART_1}
			\end{align}
			根据\emphB{偏导数三乘积法则} \eqref{EQ_CYCLIC_RELATION} 式,有
			\begin{equation}
				\myPartial{p}{T}{V} = -\myPartial{V}{T}{p} \myPartial{p}{V}{T} \semicomma
			\end{equation}
			根据\emphB{倒数关系} \eqref{EQ_RECIPROCITY_RELATION} 式,有
			\begin{equation}
				\myPartial{V}{T}{p} = \myPartial{T}{V}{p}^{-1} \fullstop
			\end{equation}
			把以上两式代入式~\eqref{EQ_C_p-C_V_WITH_MAXWELL_RELATION_PART_1},得
			\begin{align}
				C_p - C_V &= T \myPartial{p}{T}{V} \myPartial{V}{T}{p} \notag \\
				&= T \left[ -\myPartial{V}{T}{p} \myPartial{p}{V}{T} \right] \myPartial{V}{T}{p} \notag \\
				&= -T \myPartial{V}{T}{p}^2 \myPartial{V}{p}{T}^{-1} \fullstop \label{EQ_C_p-C_V_WITH_MAXWELL_RELATION_PART_2}
			\end{align}
			再把\emphB{膨胀系数}、\emphB{等温压缩系数}的定义
			\begin{braceEq}
				&\a = \frac{1}{V} \myPartial{V}{T}{p} \comma \\
				&\k_T = -\frac{1}{V} \myPartial{V}{p}{T}
			\end{braceEq}
			代入式~\eqref{EQ_C_p-C_V_WITH_MAXWELL_RELATION_PART_2},便得到
			\begin{align}
				C_p - C_V &= -T (V \a)^2 \left( -\frac{1}{V \k_T} \right) \notag \\
				&= T V \frac{\a^2}{\k_T} \label{EQ_C_p-C_V_WITH_MAXWELL_RELATION_PART_3} \fullstop
			\end{align}
			这几个量都是可测量量。热力学稳定相中,有 $\k_T > 0$。因此 $C_p$ 始终大于 $C_V$。%FIXME:20160401 后续交叉引用
			
			对于理想气体,有
			\begin{braceEq}
				&\a = \frac{1}{V} \myPartial{V}{T}{p}
				= \frac{1}{V} \left( \frac{\pd}{\pd T} \frac{n R T}{p} \right)_p
				= \frac{n R}{p V} \comma \\
				&\k_T = -\frac{1}{V} \myPartial{V}{p}{T}
				= -\frac{1}{V} \left( \frac{\pd}{\pd p} \frac{n R T}{p} \right)_T
				= \frac{n R T}{p^2 V} \fullstop
			\end{braceEq}
			根据式~\eqref{EQ_C_p-C_V_WITH_MAXWELL_RELATION_PART_3},可知
			\begin{equation}
				C_p - C_V = T V \left( \frac{n R}{p V} \right)^2 \left( \frac{p^2 V}{n R T} \right) = n R \comma
			\end{equation}
			这与式~\eqref{EQ_C_p-C_V_FOR_IDEAL_GAS} 是一致的。
		\end{myExample}
		
		\begin{myExample} \label{EG_pd_U/pd_V_WITH_FIXED_T}
			计算 $(\pd U/ \pd V)_T$。
			
			选取 $T$ 和 $V$ 作为独立变量,则
			\begin{align}
				\dd U &= T \dd S - p \dd V \notag \\
				&= T \left[ \myPartial{S}{T}{V} \dd T + \myPartial{S}{V}{T} \dd T \right] - p \dd V \notag \\
				&= T \myPartial{S}{T}{V} \dd T + \left[ T \myPartial{S}{V}{T} - p \right] \dd V \comma
			\end{align}
			因此
			\begin{align}
				\myPartial{U}{V}{T} &= T \myPartial{S}{V}{T} - p \notag \\
				&= T \myPartial{p}{T}{V} - p \fullstop \label{EQ_pd_U/pd_V_WITH_FIXED_T}
			\end{align}
			这里出现的都是状态变量或是与状态方程直接相关的量,因而也是可测量量。
			
			对于理想气体,$p V = n R T$,因此
			\begin{equation}
				\myPartial{p}{T}{V} = \left( \frac{\pd}{\pd T} \frac{n R T}{V} \right)_V = \frac{n R}{V} = \frac{p}{T} \comma
			\end{equation}
			代入式~\eqref{EQ_pd_U/pd_V_WITH_FIXED_T},可以发现
			\begin{equation}
				\myPartial{U}{V}{T} = 0 \fullstop
			\end{equation}
			这说明 $U = U(T)$,即内能 $U$ 仅是温度 $T$ 的函数,而与体积 $V$ 无关。这就是\secref{SEC_热容与焓;理想气体的性质}\subsecref{SUBSEC_理想气体的性质}中提到的利用状态方程来证明内能只与温度有关的方法。
			
			对于van der Waals气体,其状态方程 [式~\eqref{EQ_VAN_DER_WAALS_GAS_STATE_EQUATION}]为
			\begin{equation}
				\left( p + \frac{n^2 a}{V^2} \right) (V - n b) = n R T \fullstop
			\end{equation}
			在固定 $V$ 的条件下关于 $T$ 求偏导,得
			\begin{equation}
				\myPartial{p}{T}{V} (V - n b) + \left( p + \frac{n^2 a}{V^2} \right) \cdot 0 = n R \comma
			\end{equation}
			即
			\begin{equation}
				\myPartial{p}{T}{V} = \frac{n R}{V - n b} \fullstop
			\end{equation}
			因此
			\begin{equation}
				\myPartial{U}{V}{T} = \frac{n R T}{V - n b} - p = \frac{n^2 a}{V^2} \fullstop
			\end{equation}
			可见van der Waals气体的内能与体积\emphB{有关}。%TODO:20160401 与温度T有关?
		\end{myExample}
		
		\begin{myExample}
			证明绝热压缩系数与等温压缩系数之比
			\begin{equation}
				\frac{\k_\text{a}}{\k_T} = \frac{C_V}{C_p} \fullstop
			\end{equation}%HACK:20160401 绝热过程下标用\text{a}
			
			可逆的绝热过程是等熵过程,因此把 $\k_\text{a}$ 的下标改用“$S$”。%TODO:20160401 为什么可逆
			
			根据定义
			\begin{braceEq}
				\k_S &\eqdef -\frac{1}{V} \myPartial{V}{p}{S} \comma \\
				\k_T &\eqdef -\frac{1}{V} \myPartial{V}{p}{T} \fullstop
			\end{braceEq}
			因此
			\begin{align}
				\frac{\k_S}{\k_T} &= \left. \myPartialDisplay{V}{p}{S} \, \middle/ \, \myPartialDisplay{V}{p}{T} \right. \notag \\
				&= \left. \frac{ \pd \; (V, \, S) }{ \pd \; (p, \, S) } \, \middle/ \, \frac{ \pd \; (V, \, T) }{ \pd \; (p, \, T) } \right. \notag \\
				&= \left. \frac{ \pd \; (V, \, S) }{ \pd \; (V, \, T) } \, \middle/ \, \frac{ \pd \; (p, \, S) }{ \pd \; (p, \, T) } \right. \myTag{见式~\eqref{}} \\%FIXME:20160402 交叉引用
				&= \left. \myPartialDisplay{S}{T}{V} \, \middle/ \, \myPartialDisplay{S}{T}{p} \right. \notag \\
				&= \frac{C_V / T}{C_p / T} \myTag{见\egref{EG_C_p-C_V}} \\
				&= \frac{C_V}{C_p} \fullstop
			\end{align}
			
			\begin{myProof}
				这里证明几个上面用到的结论:%TODO:偏微分关系的证明
				\begin{myEnum2}
					\item
						\begin{equation} \label{EQ_PARTIAL_DIFFERENTIAL_TO_JACOBI_DET}
							\myPartial{f}{g}{h} = \frac{ \pd \; (f, \, h) }{ \pd \; (g, \, h) } \fullstop
						\end{equation}
					\item
						\begin{equation} \label{EQ_JACOBI_DET_FRACTION}
							\myPartial{f}{g}{h} = \left. \frac{ \pd \; (f, \, h) }{ \pd \; (x, \, y) } \, \middle/ \, \frac{ \pd \; (g, \, h) }{ \pd \; (x, \, y) } \right. \fullstop
						\end{equation}
					\item
						\begin{equation} \label{EQ_INVERSE_JACOBI_DET}
							\frac{ \pd \; (f, \, g) }{ \pd \; (x, \, y) } = \left[ \frac{ \pd \; (x, \, y) }{ \pd \; (f, \, g) } \right]^{-1} \fullstop
						\end{equation}
				\end{myEnum2}
			\end{myProof}
		\end{myExample}
		
\section{Joule-Thomson效应;绝热膨胀与制冷}%FIXME:连字符kerning
	\subsection{节流过程}
		为了研究气体的内能,Joule先采用了气体自由膨胀的方法[见\secref{SEC_热容与焓;理想气体的性质}\subsecref{SUBSEC_理想气体的性质}],但其精度不佳。之后(1852年),他又与Thomson采取了另外的手段,即\emphA{节流过程}。
		
		如图,%TODO:20160402 图片,节流过程
		在一根绝热管中间放置一个多孔塞,使气体无法很快地通过。保持多孔塞两边的压强差,则气体将从其一边不断地流到另一边,该过程就是\emphB{节流过程}。
		
		开始时,气体完全在多孔塞的一边。设其压强、体积、内能分别为 $p_1$、$V_1$ 和 $U_1$。让外界做功,直到气体完全流到多孔塞的另一边。此时,气体的压强、体积、内能分别为 $p_2$、$V_2$ 和 $U_2$。显然,外界做功
		\begin{equation}
			\db W = p_1 V_1 - p_2 V_2 \semicomma
		\end{equation}
		因为是绝热过程,因此 $\db Q = 0$。根据热力学第一定律,
		\begin{equation}
			U_2 - U_1 = \db W + \db Q = p_1 V_1 - p_2 V_2 \comma
		\end{equation}
		即
		\begin{equation}
			U_1 + p_1 V_1 = U_2 + p_2 V_2 \fullstop
		\end{equation}
		这说明节流过程是一个\emphB{等焓过程},即 $H_1 = H_2$。\footnote{
			要注意该过程是一个\emphB{不可逆过程}。
		}
		
		定义\emphA{Joule-Thomson系数}
		\begin{equation}
			\m \eqdef \myPartial{T}{p}{H} \fullstop
		\end{equation}
		根据\emphB{偏导数三乘积法则} \eqref{EQ_CYCLIC_RELATION} 式,可知
		\begin{equation}
			\m = -\myPartial{H}{p}{T} \myPartial{T}{H}{p} \fullstop
			%TODO:20160402 没写,制冷曲线等
		\end{equation}
		
	\subsection{绝热膨胀过程}
		对于绝热过程,选取 $T$ 和 $p$ 作为独立变量,则
		\begin{equation}
			\dd S = \myPartial{S}{T}{p} \dd T+ \myPartial{S}{p}{T} \dd p = 0 \fullstop
		\end{equation}
		因此
		\begin{align}
			\myPartial{T}{p}{S} &= -\myPartial{S}{p}{T} \myPartial{T}{S}{p} \myTag{偏导数三乘积法则} \\
			&= -\left. \myPartialDisplay{S}{p}{T} \, \middle/ \, \myPartialDisplay{S}{T}{p} \right. \myTag{倒数关系} \\
			&= -\frac{T}{C_p} \myPartial{S}{p}{T} \myTag{见式~\eqref{EQ_C_p_IN_T_AND_S}} \\
			&= \frac{T}{C_p} \myPartial{V}{T}{p} \myTag{Maxwell关系} \\
			&= \frac{T V \a}{C_p} > 0  \myTagNumbering{见式~\eqref{EQ_DEFINITION_OF_EXPANSION_COEFFICIENT}} \fullstop
		\end{align}
		随着压强的增大,分子间距离减小,相互作用力增大,因而总能量减少,分子平均动能增加,从而使得温度上升。%TODO:20160402 解释对不对——能量
	\subsection{低温的实现}
	
\section{热力学函数的确定}
\section{特性函数}
\section{热辐射的热力学理论}
\raggedbottom%FIXME:20160315 交叉引用、脚注每页重新计数失效,必须加上该行
\pagebreak